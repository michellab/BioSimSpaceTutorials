%%%%%%%%%%%%%%%%%%%%%%%%%%%%%%%%%%%%%%%%%%%%%%%%%%%%%%%%%%%%
%%% LIVECOMS ARTICLE TEMPLATE FOR BEST PRACTICES GUIDE
%%% ADAPTED FROM ELIFE ARTICLE TEMPLATE (8/10/2017)
%%%%%%%%%%%%%%%%%%%%%%%%%%%%%%%%%%%%%%%%%%%%%%%%%%%%%%%%%%%%
%%% PREAMBLE
\documentclass[9pt,tutorial]{livecoms}
% Use the 'onehalfspacing' option for 1.5 line spacing
% Use the 'doublespacing' option for 2.0 line spacing
% Use the 'lineno' option for adding line numbers.
% Use the "ASAPversion' option following article acceptance to add the DOI and relevant dates to the document footer.
% Use the 'pubversion' option for adding the citation and publication information to the document footer, when the LiveCoMS issue is finalized.
% The 'bestpractices' option for indicates that this is a best practices guide.
% Omit the bestpractices option to remove the marking as a LiveCoMS paper.
% Please note that these options may affect formatting.

\usepackage{lipsum} % Required to insert dummy text
\usepackage[version=4]{mhchem}
\usepackage{siunitx}
\DeclareSIUnit\Molar{M}
\usepackage[italic]{mathastext}
\graphicspath{{figures/}}

%%%%%%%%%% USER INPUT PACKAGES & FUNCTIONS
\usepackage{listings}
\lstset{
	basicstyle=\ttfamily,
	commentstyle={},
	breakatwhitespace=true,
	breaklines=true,
	language=bash
}
\usepackage{pythonhighlight}

%%%%%%%%%%%%%%%%%%%%%%%%%%%%%%%%%%%%%%%%%%%%%%%%%%%%%%%%%%%%
%%% IMPORTANT USER CONFIGURATION
%%%%%%%%%%%%%%%%%%%%%%%%%%%%%%%%%%%%%%%%%%%%%%%%%%%%%%%%%%%%

\newcommand{\versionnumber}{1.0}  % you should update the minor version number in preprints and major version number of submissions.
\newcommand{\githubrepository}{\url{https://github.com/michellab/BioSimSpaceTutorials}}  %this should be the main github repository for this article

%%%%%%%%%%%%%%%%%%%%%%%%%%%%%%%%%%%%%%%%%%%%%%%%%%%%%%%%%%%%
%%% ARTICLE SETUP
%%%%%%%%%%%%%%%%%%%%%%%%%%%%%%%%%%%%%%%%%%%%%%%%%%%%%%%%%%%%
\title{A Suite of Tutorials for the BioSimSpace framework for interoperable biomolecular simulation [Article v\versionnumber]}
% Everyone in alphabetical order other than Lester

\author[1*]{Lester O. Hedges}
\author[]{Sofia Bariami}
\author[2]{Adele Hardie}
\author[3]{Dominykas Lukauskis}
\author[2]{Antonia S.J.S. Mey}
\author[2*]{Julien Michel}
\author[2\authfn{1}]{Jenke Scheen}

%\author[1,2\authfn{1}\authfn{3}]{Firstname Middlename Familyname}
%\author[2\authfn{1}\authfn{4}]{Firstname Initials Surname}
%\author[2*]{Firstname Surname}
\affil[1]{Advanced Computing Research Centre, University of Bristol, UK}
\affil[2]{EaStCHEM School of
Chemistry, University of Edinburgh, UK}
\affil[3]{Department of Chemistry and Institute of Structural and Molecular Biology, University College
London, UK}


\corr{julien.michel@ed.ac.uk}{JM}  % Correspondence emails.  FMS and FS are the appropriate authors initials.
\corr{lester.hedges@bristol.ac.uk }{LH}

\orcid{Lester Hedges}{EEEE-FFFF-GGGG-HHHH}
\orcid{Adele Hardie}{EEEE-FFFF-GGGG-HHHH}
\orcid{Dominykas Lukauskis}{EEEE-FFFF-GGGG-HHHH}
\orcid{Antonia Mey}{0000-0001-7512-5252}
\orcid{Julien Michel}{0000-0003-0360-1760}
\orcid{Jenke Scheen}{EEEE-FFFF-GGGG-HHHH}

%\contrib[\authfn{1}]{These authors contributed equally to this work}
%\contrib[\authfn{2}]{These authors also contributed equally to this work}

\presentadd[\authfn{1}]{Computational and Systems Biology Program, Sloan Kettering Institute, Memorial Sloan Kettering Cancer Center, New York NY, USA}
%\presentadd[\authfn{4}]{Department, Institute, Country}

\blurb{This LiveCoMS document is maintained online on GitHub at \githubrepository; to provide feedback, suggestions, or help improve it, please visit the GitHub repository and participate via the issue tracker.}

%%%%%%%%%%%%%%%%%%%%%%%%%%%%%%%%%%%%%%%%%%%%%%%%%%%%%%%%%%%%
%%% PUBLICATION INFORMATION
%%% Fill out these parameters when available
%%% These are used when the "pubversion" option is invoked
%%%%%%%%%%%%%%%%%%%%%%%%%%%%%%%%%%%%%%%%%%%%%%%%%%%%%%%%%%%%
\pubDOI{10.XXXX/YYYYYYY}
\pubvolume{<volume>}
\pubissue{<issue>}
\pubyear{<year>}
\articlenum{<number>}
\datereceived{Day Month Year}
\dateaccepted{Day Month Year}

%%%%%%%%%%%%%%%%%%%%%%%%%%%%%%%%%%%%%%%%%%%%%%%%%%%%%%%%%%%%
%%% ARTICLE START
%%%%%%%%%%%%%%%%%%%%%%%%%%%%%%%%%%%%%%%%%%%%%%%%%%%%%%%%%%%%

\begin{document}

\begin{frontmatter}
\maketitle

\begin{abstract}
This tutorial serves as a getting-started guide for BioSimSpace (BSS), an interoperable molecular dynamics framework, that allows simulations with different sets of molecular dynamics software packages. The tutorial will cover four main use cases for BioSimSpace. The introductory tutorial introduces the basic structure of BioSimSpace, how to use the API to access functionality, and how to write to use it for setting up and running standard molecular dynamics simulations. Furthermore, the tutorial provides three advanced use cases of BSS on how to set up and run a funnel molecular dynamics simulation, steered MD in combination with Markov State modelling, and alchemical free energy calculations. 
\end{abstract}

\end{frontmatter}




\section{Introduction}

Here you would explain what problem you are tackling and briefly motivate your work.

In this particular template, we have removed most of the usage examples which occur in \texttt{sample-document.tex} to provide a minimal template you can modify; however, we retain a couple of examples illustrating more unusual features of our templates/article class, such as the checklists, and information on algorithms and pseudocode.

Keep in mind, as you prepare your manuscript, that you should plan for a representative image  which will be used to highlight your article on the journal website and publications. Usually, this would be one of your figures, but it must also be uploaded separately upon article submission. We give specific guidelines for this image on the journal website in the section on article submission (see \url{https://livecomsjournal.github.io/authors/policies/index.html#article-submission}).

Additionally, for well-formatted manuscripts, we recommend that you let LaTeX handle figure/table placement for you as much as possible, so please avoid specifying strenuous float instructions like `[h!]` and `[H]` as much as possible.

\subsection{Scope}

Tutorials should endeavor to cover the specific task at hand, and also highlight how the steps might need to be modified (or additional care might need to be taken at particular points) to handle more general cases.

The scope of the tutorial, as well as the expected proficiencies / outcomes for researchers who complete the tutorial, should be clearly defined.
This will often happen in a specific section or subsection in the article itself.

\section{Prerequisites}

Here you would identify prerequisites/background knowledge that are assumed by your work, as well as any software/license requirements.

\subsection{Background knowledge}
Tutorials should clearly define what concepts or abilities researchers will need to complete the tutorial (e.g., some proficiency in Python; experience with Jupyter notebooks; knowledge of classical MD; etc).

\subsection{Software/system requirements}
Tutorials should clearly define what system and/or software requirements the researcher will need to complete the tutorial (e.g., VMD version 1.9 or newer, AMBER, etc.). Tutorials requiring specific software packages must provide instructions and files for the referenced version of the software.

\section{Content and links}

A tutorial will normally draw on additional files and materials; clearly indicate where and how these are available, with links, and how they are being archived for the long-term and maintained so they stay current.
You will likely want to reference your GitHub repository as a central point to access all of this information, and then the GitHub repository may link out to other content as needed.

%TODO Fix layout issues 
\subsection{Tutorial 1: Introduction}
Author: Lester Hedges Email:~~ lester.hedges@bristol.ac.uk

\hypertarget{biosimspace}{%
\section{BioSimSpace}\label{biosimspace}}

The companion notebook for this section can be found
\href{https://github.com/michellab/BioSimSpaceTutorials/blob/4844562e7d2cd0b269cead56562ec16a3dfaef7c/01_introduction/01_introduction.ipynb}{here}.

\hypertarget{introduction}{%
\subsection{Introduction}\label{introduction}}

Welcome to this workshop on \href{https://biosimspace.org}{BioSimSpace},
an \emph{interoperable} Python framework for biomolecular simulation. In
this introductory session you will learn:

\begin{itemize}
\tightlist
\item
  What are the key concepts behind BioSimSpace.
\item
  How to set up molecular systems ready for simulation.
\item
  How to configure and run a range of molecular dynamics protocols using
  different simulation engines.
\item
  How to write interoperable worflow components and run them in a
  variety of ways.
\end{itemize}

\hypertarget{what-is-biosimspace}{%
\subsection{What is BioSimSpace?}\label{what-is-biosimspace}}

As a computational chemist you are likely overwhelmed by the amount of
different software packages that are available to you. Having choice is
a good thing, but too much can become a burden. I'm sure you have all
come across at least one of the following:

\begin{itemize}
\tightlist
\item
  I know how to solve the problem with package X but I want to use
  package Y.
\item
  How can I share my script with a collaborator who doesn't use the same
  software stack?
\item
  How can I take advantage of the best tool for the job for different
  parts of my workflow?
\item
  How can I compare methodology / results between simulation engines?
\end{itemize}

Solving these problems is the core goal of BioSimSpace. The wealth of
fantastic software in our community is a real asset but
\emph{interoperability} is currently a problem. Since there is no point
reinventing the wheel, BioSimSpace is not an attempt to produce yet
another molecular simulation package that reproduces all of the
functionality from existing programs. This would result in just another
tool for you to learn, along with yet another set of standards and
formats. Instead, BioSimSpace is essentially just a set of \emph{shims},
or bits of \emph{glue}, that connect together existing software
packages, allowing you to interact with them using a consistent Python
interface.

\hypertarget{why-biosimspace}{%
\subsection{Why BioSimSpace?}\label{why-biosimspace}}

By using BioSimSpace you will be less reliant on the use of brittle
scripts to connect different software packages together. BioSimSpace
builds on top of existing and open Python tools within the biomolecular
community, e.g. \href{https://www.rdkit.org/}{RDKit},
\href{http://openmm.org/}{OpenMM},
\href{https://github.com/openforcefield/openff-toolkit}{Open Force
Field}. As such, you are able to leverage the power of other packages,
with which you may already be familiar, and to mix-and-match
functionality where required.

With BioSimSpace you will be able to:

\begin{itemize}
\tightlist
\item
  Write generic workflow components \emph{once} in a package-agnostic
  language.
\item
  Run the same script from the command-line, Jupyter, or within a
  workflow engine.
\item
  Use the most suitable package that is availabe on your computer.
\item
  Continue using your favourite package X but be able to share scripts
  with your collaborator who prefers package Y.
\item
  Be able to take advantage of new software packages and hardware
  resources as and when they become available.
\end{itemize}

\hypertarget{what-can-biosimspace-do}{%
\subsection{What can BioSimSpace do?}\label{what-can-biosimspace-do}}

BioSimSpace provides a suite of packages with a range of different
functionality.

At present:

\begin{itemize}
\tightlist
\item
  File conversion:
  \href{https://biosimspace.org/api/index_IO.html}{BioSimSpace.IO}
\item
  Parameterisation:
  \href{https://biosimspace.org/api/index_Parameters.html}{BioSimSpace.Parameters}
\item
  Solvation:
  \href{https://biosimspace.org/api/index_Solvent.html}{BioSimSpace.Solvent}
\item
  Molecular dynamics:
  \href{https://biosimspace.org/api/index_Protocol.html}{BioSimSpace.Protocol},
  \href{https://biosimspace.org/api/index_Process.html}{BioSimSpace.Process},
  \href{https://biosimspace.org/api/index_MD.html}{BioSimSpace.MD}
\item
  Free-energy perturbation:
  \href{https://biosimspace.org/api/index_Align.html}{BioSimSpace.Align},
  \href{https://biosimspace.org/api/index_FreeEnergy.html}{BioSimSpace.FreeEnergy}
\item
  Metadynamics:
  \href{https://biosimspace.org/api/index_Metadynamics.html}{BioSimSpace.Metadynamics}
\item
  Trajectory handling:
  \href{https://biosimspace.org/api/index_Trajectory.html}{BioSimSpace.Trajectory}
\item
  Interactive visualisation:
  \href{https://biosimspace.org/api/index_Notebook.html}{BioSimSpace.Notebook}
\item
  Workflow components:
  \href{https://biosimspace.org/api/index_Gateway.html}{BioSimSpace.Gateway}
\end{itemize}

\hypertarget{key-concepts}{%
\subsection{Key concepts}\label{key-concepts}}

Before getting started it's worth spending a little time covering a few
of the key concepts of BioSimSpace.

\hypertarget{file-conversion}{%
\subsubsection{File conversion}\label{file-conversion}}

While, broadly speaking, the different molecular dynamics engines offer
a similar range of features, their interfaces are quite different. This
makes it hard to take expertise in one package and immediately apply it
to another. At the heart of this problem is the incompatibility between
the molecular file formats used by the different packages. While they
all contain the same information, i.e.~how atoms are laid out in space
and how they interact with each other, the structure of the files is
very different. In order to provide interoperability betwen packages we
will need to be able to read and write many different file formats, and
be able to interconvert between them too.

Let's import the BioSimSpace Python package and see what we can do. For
convenience, we'll rename the package to BSS to save us typing:

\begin{Shaded}
\begin{Highlighting}[]
\ImportTok{import}\NormalTok{ BioSimSpace }\ImportTok{as}\NormalTok{ BSS}
\end{Highlighting}
\end{Shaded}

To see what file formats are supported by BioSimSpace, execute the cell
below.

\begin{Shaded}
\begin{Highlighting}[]
\NormalTok{BSS.IO.fileFormats()}
\end{Highlighting}
\end{Shaded}

\begin{verbatim}
['Gro87', 'GroTop', 'MOL2', 'PDB', 'PRM7', 'PSF', 'RST', 'RST7']
\end{verbatim}

Note that these refer to specific file \emph{formats}, rather than file
\emph{extensions}. BioSimSpace doesn't care about file extensions, it's
the \emph{contents} of the file that's important.

If you aren't familiar with a particular format, you can get more
information as follows, e.g.:

\begin{Shaded}
\begin{Highlighting}[]
\NormalTok{BSS.IO.formatInfo(}\StringTok{"GroTop"}\NormalTok{)}
\end{Highlighting}
\end{Shaded}

\begin{verbatim}
'Gromacs Topology format files.'
\end{verbatim}

The \texttt{BSS.IO.readMolecules} function is used to read molecular
information from file. We've provided some example input files for you
in the \texttt{inputs} directory. Let's take a look at some of these.

\begin{Shaded}
\begin{Highlighting}[]
\OperatorTok{!}\NormalTok{ls inputs}
\end{Highlighting}
\end{Shaded}

\begin{verbatim}
1jr5.crd  1jr5.top  ala.crd  kigaki.gro  methanol.pdb
1jr5.pdb  2JJC.pdb  ala.top  kigaki.top
\end{verbatim}

The \texttt{ala.crd} and \texttt{ala.top} files define a solvated
alanine dipeptide system in AMBER format. Execute the cell below to see
part of the topology file:

\begin{Shaded}
\begin{Highlighting}[]
\OperatorTok{!}\NormalTok{head }\OperatorTok{-}\NormalTok{n }\DecValTok{20}\NormalTok{ inputs}\OperatorTok{/}\NormalTok{ala.top}
\end{Highlighting}
\end{Shaded}

\begin{verbatim}
%VERSION  VERSION_STAMP = V0001.000  DATE = 06/30/15  11:44:23                  
%FLAG TITLE                                                                     
%FORMAT(20a4)                                                                   
ACE                                                                             
%FLAG POINTERS                                                                  
%FORMAT(10I8)                                                                   
    1912       9    1902       9      25      11      43      24       0       0
    2619     633       9      11      24      13      21      20      10       1
       0       0       0       0       0       0       0       1      10       0
       0
%FLAG ATOM_NAME                                                                 
%FORMAT(20a4)                                                                   
HH31CH3 HH32HH33C   O   N   H   CA  HA  CB  HB1 HB2 HB3 C   O   N   H   CH3 HH31
HH32HH33O   H1  H2  O   H1  H2  O   H1  H2  O   H1  H2  O   H1  H2  O   H1  H2  
O   H1  H2  O   H1  H2  O   H1  H2  O   H1  H2  O   H1  H2  O   H1  H2  O   H1  
H2  O   H1  H2  O   H1  H2  O   H1  H2  O   H1  H2  O   H1  H2  O   H1  H2  O   
H1  H2  O   H1  H2  O   H1  H2  O   H1  H2  O   H1  H2  O   H1  H2  O   H1  H2  
O   H1  H2  O   H1  H2  O   H1  H2  O   H1  H2  O   H1  H2  O   H1  H2  O   H1  
H2  O   H1  H2  O   H1  H2  O   H1  H2  O   H1  H2  O   H1  H2  O   H1  H2  O   
H1  H2  O   H1  H2  O   H1  H2  O   H1  H2  O   H1  H2  O   H1  H2  O   H1  H2  
\end{verbatim}

Let's now read the molecules from file. The
\texttt{BSS.IO.readMolecules} function automatically
\href{https://en.wikipedia.org/wiki/Glob_(programming)}{globs} the
passed string, so wildcard matching can be used to determine the files.
(Here the asterisk matches any characters, i.e.~we are reading
\emph{all} files in the \texttt{inputs} directory with \texttt{ala} as
the file prefix.)

\begin{Shaded}
\begin{Highlighting}[]
\NormalTok{system }\OperatorTok{=}\NormalTok{ BSS.IO.readMolecules(}\StringTok{"inputs/ala.*"}\NormalTok{)}
\end{Highlighting}
\end{Shaded}

N.B. We could have explictly specified each file using a list of
strings.

Note that we don't have to specify anything about the file format.
BioSimSpace actually reads the files with \emph{all} of its inbuilt
parsers in parallel. If a parser fails to read the file then it is
immediately rejected and we move on to the next. As long as all of the
files that are read contain a consistent topology, then BioSimSpace will
be able to read them.

To see what file formats are associated with the system, run:

\begin{Shaded}
\begin{Highlighting}[]
\NormalTok{system.fileFormat()}
\end{Highlighting}
\end{Shaded}

\begin{verbatim}
'PRM7,RST7'
\end{verbatim}

As expected, BioSimpace has detected that these were AMBER format
topology and coordinate files.

The molecules are now loaded into a \texttt{System} object. BioSimSpace
objects are thin wrappers around the equivalent objects from
\href{https://github.com/michellab/Sire}{Sire}. For those that are
familiar with Sire, you can get always access the underlying object
directly using \texttt{system.\_sire\_object}. This \emph{private}
member is hidden from the user. Sire provides an extremely powerful and
flexible set of tools for molecular manipulation and editing. While
BioSimSpace directly exposes only a small subset of this functionality
to the user, the full power of Sire is always easily available when
needed.

We can query how many molecules there are as follows:

\begin{Shaded}
\begin{Highlighting}[]
\NormalTok{system.nMolecules()}
\end{Highlighting}
\end{Shaded}

\begin{verbatim}
631
\end{verbatim}

To see how many water molecules there are:

\begin{Shaded}
\begin{Highlighting}[]
\NormalTok{system.nWaterMolecules()}
\end{Highlighting}
\end{Shaded}

\begin{verbatim}
630
\end{verbatim}

To search for all nitrogen atoms in residues named \texttt{ALA}:

\begin{Shaded}
\begin{Highlighting}[]
\NormalTok{system.search(}\StringTok{"element N and resname ALA"}\NormalTok{)}
\end{Highlighting}
\end{Shaded}

\begin{verbatim}
<BioSimSpace.SearchResult: nResults=1>
\end{verbatim}

Create a new system using from the first 10 and last molecule in the
system:

\begin{Shaded}
\begin{Highlighting}[]
\NormalTok{new_system }\OperatorTok{=}\NormalTok{ (system[:}\DecValTok{10}\NormalTok{] }\OperatorTok{+}\NormalTok{ system[}\OperatorTok{-}\DecValTok{1}\NormalTok{]).toSystem()}
\end{Highlighting}
\end{Shaded}

N.B. Information from the topology file pertaining to the molecular
force field are stored internally as computer algebra expressions,
allowing us to mathematically interconvert between different
representations of the terms when writing to a different format. If
desired, you could even edit the system to create your own unique force
field parameters, although this beyond the scope of this tutorial.

Now that we have a molecular system, let's write it back to disk in a
different format. Unsurprisingly, this is done using the
\texttt{BSS.IO.saveMolecules} function. Execute the cell below to write
the system to GROMACS format coordinate and topology files.

\begin{Shaded}
\begin{Highlighting}[]
\NormalTok{BSS.IO.saveMolecules(}\StringTok{"ala"}\NormalTok{, system, [}\StringTok{"Gro87"}\NormalTok{, }\StringTok{"GroTop"}\NormalTok{])}
\end{Highlighting}
\end{Shaded}

\begin{verbatim}
['/home/lester/Code/BioSimSpaceTutorials/01_introduction/ala.gro',
 '/home/lester/Code/BioSimSpaceTutorials/01_introduction/ala.top']
\end{verbatim}

Run the cell below to examine the start of the GROMACS topology file.

\begin{Shaded}
\begin{Highlighting}[]
\OperatorTok{!}\NormalTok{head }\OperatorTok{-}\NormalTok{n }\DecValTok{20}\NormalTok{ ala.top}
\end{Highlighting}
\end{Shaded}

\begin{verbatim}
; Gromacs Topology File written by Sire
; File written 05/06/21  14:17:35
[ defaults ]
; nbfunc      comb-rule       gen-pairs      fudgeLJ     fudgeQQ
  1           2               yes            0.5         0.833333

[ atomtypes ]
; name      at.num        mass      charge   ptype       sigma     epsilon
     C           6   12.010700    0.000000       A    0.339967    0.359824
    CT           6   12.010700    0.000000       A    0.339967    0.457730
    CX           6   12.010700    0.000000       A    0.339967    0.457730
     H           1    1.007940    0.000000       A    0.106908    0.065689
    H1           1    1.007940    0.000000       A    0.247135    0.065689
    HC           1    1.007940    0.000000       A    0.264953    0.065689
    HW           1    1.007940    0.000000       A    0.000000    0.000000
     N           7   14.006700    0.000000       A    0.325000    0.711280
     O           8   15.999400    0.000000       A    0.295992    0.878640
    OW           8   15.999400    0.000000       A    0.315061    0.636386

[ moleculetype ]
\end{verbatim}

\hypertarget{topology-preservation}{%
\subsubsection{Topology preservation}\label{topology-preservation}}

One of the other pitfalls of working with different molecular simulation
engines is that they often have quirks regarding naming conventions and
atom ordering. This means that what you get back from a given program
might not be the same as what you put in, making it tricky to
cross-reference parts of the system that are of specific interest.

N.B. Differerent naming conventions are one of the hardest problems with
interoperability.

BioSimSpace tries to preserve the intial molecular topology during any
interaction with external tools so that it can be used as a consistent
reference. For example, while we might need to rename the water topology
to match the conventions of a particular molecular dynamics engine, we
always copy the updated coordinates from a simulation back into the
original system so that the naming that the user gets back is unchanged.

Another common topology feature that can be lost is chain identifiers.
For example, a \href{https://www.rcsb.org/}{Protein Data Bank} file
might contain labels for chains, but these would be lost during
parameterisation with the
\href{https://ambermd.org/AmberTools.php}{AmberTools} suite since it has
no internal concept of chains.

As an example, consider the following PDB file:

\begin{Shaded}
\begin{Highlighting}[]
\NormalTok{system_pdb }\OperatorTok{=}\NormalTok{ BSS.IO.readMolecules(}\StringTok{"inputs/1jr5.pdb"}\NormalTok{)}
\end{Highlighting}
\end{Shaded}

This is read as a single molecule containing two chains:

\begin{Shaded}
\begin{Highlighting}[]
\BuiltInTok{print}\NormalTok{(}\SpecialStringTok{f"mols = }\SpecialCharTok{\{}\NormalTok{system_pdb}\SpecialCharTok{.}\NormalTok{nMolecules()}\SpecialCharTok{\}}\SpecialStringTok{, chains = }\SpecialCharTok{\{}\NormalTok{system_pdb}\SpecialCharTok{.}\NormalTok{nChains()}\SpecialCharTok{\}}\SpecialStringTok{"}\NormalTok{)}
\end{Highlighting}
\end{Shaded}

\begin{verbatim}
mols = 1, chains = 2
\end{verbatim}

The molecule in the system has a topology but no force field
information, as we can see by querying the \emph{properties} of the
underlying Sire object:

\begin{Shaded}
\begin{Highlighting}[]
\NormalTok{system_pdb[}\DecValTok{0}\NormalTok{]._sire_object.propertyKeys()}
\end{Highlighting}
\end{Shaded}

\begin{verbatim}
['alt_loc',
 'beta_factor',
 'coordinates',
 'element',
 'insert_code',
 'formal_charge',
 'fileformat',
 'occupancy',
 'is_het']
\end{verbatim}

After parameterising the molecule with an AMBER protein force field, we
end up with the following system:

\begin{Shaded}
\begin{Highlighting}[]
\NormalTok{system_amber }\OperatorTok{=}\NormalTok{ BSS.IO.readMolecules([}\StringTok{"inputs/1jr5.crd"}\NormalTok{, }\StringTok{"inputs/1jr5.top"}\NormalTok{])}
\end{Highlighting}
\end{Shaded}

In contrast, this is read as a two molecules with zero chains:

\begin{Shaded}
\begin{Highlighting}[]
\BuiltInTok{print}\NormalTok{(}\SpecialStringTok{f"mols = }\SpecialCharTok{\{}\NormalTok{system_amber}\SpecialCharTok{.}\NormalTok{nMolecules()}\SpecialCharTok{\}}\SpecialStringTok{, chains = }\SpecialCharTok{\{}\NormalTok{system_amber}\SpecialCharTok{.}\NormalTok{nChains()}\SpecialCharTok{\}}\SpecialStringTok{"}\NormalTok{)}
\end{Highlighting}
\end{Shaded}

\begin{verbatim}
mols = 2, chains = 0
\end{verbatim}

Molecules in the parameterised system contain a different set of
properties, including those pertaining to the force field.

\begin{Shaded}
\begin{Highlighting}[]
\NormalTok{system_amber[}\DecValTok{0}\NormalTok{]._sire_object.propertyKeys()}
\end{Highlighting}
\end{Shaded}

\begin{verbatim}
['intrascale',
 'bond',
 'angle',
 'fileformat',
 'atomtype',
 'LJ',
 'charge',
 'connectivity',
 'forcefield',
 'mass',
 'coordinates',
 'treechain',
 'gb_radii',
 'ambertype',
 'improper',
 'element',
 'parameters',
 'gb_screening',
 'gb_radius_set',
 'dihedral']
\end{verbatim}

If we want to preserve the topology of the original molecule, yet add
the updated properties from the molecules in the new system, then we can
make it \emph{compatible}. BioSimSpace provides functionality to do this
for you:

\begin{Shaded}
\begin{Highlighting}[]
\CommentTok{# Extract the original molecule.}
\NormalTok{new_mol }\OperatorTok{=}\NormalTok{ system_pdb[}\DecValTok{0}\NormalTok{]}

\CommentTok{# Make it compatible with the new system, i.e. add new properties while}
\CommentTok{# preserving the topology.}
\NormalTok{new_mol.makeCompatibleWith(system_amber)}
\end{Highlighting}
\end{Shaded}

N.B. This is done implicitly whenever calling any BioSimSpace
parameterisation function.

Let's now check the number of chains in the new molecule, as well as the
properties that are associated with it.

\begin{Shaded}
\begin{Highlighting}[]
\BuiltInTok{print}\NormalTok{(}\SpecialStringTok{f"chains = }\SpecialCharTok{\{}\NormalTok{new_mol}\SpecialCharTok{.}\NormalTok{nChains()}\SpecialCharTok{\}}\SpecialStringTok{"}\NormalTok{)}
\NormalTok{new_mol._sire_object.propertyKeys()}
\end{Highlighting}
\end{Shaded}

\begin{verbatim}
chains = 2





['formal_charge',
 'is_het',
 'intrascale',
 'bond',
 'angle',
 'fileformat',
 'atomtype',
 'LJ',
 'charge',
 'connectivity',
 'forcefield',
 'alt_loc',
 'mass',
 'beta_factor',
 'coordinates',
 'occupancy',
 'treechain',
 'gb_radii',
 'ambertype',
 'improper',
 'element',
 'insert_code',
 'gb_screening',
 'gb_radius_set',
 'dihedral']
\end{verbatim}

Author: Lester Hedges Email:~~ lester.hedges@bristol.ac.uk

\hypertarget{molecular-setup}{%
\section{Molecular setup}\label{molecular-setup}}

The companion notebook for this section can be found
\href{https://github.com/michellab/BioSimSpaceTutorials/blob/4844562e7d2cd0b269cead56562ec16a3dfaef7c/01_introduction/02_molecular_setup.ipynb}{here}

\hypertarget{introduction}{%
\subsection{Introduction}\label{introduction}}

In this section we will learn how to use BioSimSpace to set up a
molecular system ready for simulation. Starting from a molecular
topology in the form of a \href{https://www.rcsb.org/}{Protein Data
Bank} format file, we will learn how to parameterise molecules using
different molecular
\href{https://en.wikipedia.org/wiki/Force_field_(chemistry)}{force
fields}, then solvate them using various
\href{https://en.wikipedia.org/wiki/Water_model}{water models}.

\hypertarget{a-note-regarding-molecular-input}{%
\subsubsection{A note regarding molecular
input}\label{a-note-regarding-molecular-input}}

The starting point for many simulations is a molecular topology in the
form of a \href{https://www.rcsb.org/}{PDB} file. This file contains
information regarding the structure of the molecule (its constituent
residues and atoms), the layout of atoms in space (in the form of 3D
atomic coordinates), and sometimes additional molecular information such
as the formal charge of each atom. What this file does not contain is
information describing how the atoms in the molecule \emph{interact},
i.e.~what are the functional forms and parameters for the terms in the
molecular potential. This file is then used as the input to a
\emph{parameterisation engine}, which typically matches the atoms and
residues against templates in order to \emph{parameterise} the molecule
with a chosen force field. As such, the accuracy of the original
topology is of critical importance: Atoms and residues \emph{must} have
the correct names, and the topology \emph{must} be complete, i.e.~no
missing atoms.

Unfortunately, many tools do a poor job in preparing PDB files,
e.g.~having quirks with their naming conventions, excluding certain
atoms, etc. Since it is impossible to account for all such
inconsistencies, which often takes detailed knowledge of the particular
system and tool in question, BioSimSpace takes the approach that the
original files used to create a starting moleular system should be
properly formatted from the outset. We don't want to make guesses as to
what the user intended, or leave them confused if unexpected behaviour
occurs later down the line.

If pre-processing of the PDB file is required, then we recommend using
one of the following third-party tools:

\begin{itemize}
\tightlist
\item
  \href{https://github.com/Amber-MD/pdb4amber}{pdb4amber}
\item
  \href{https://htmlpreview.github.io/?https://github.com/openmm/pdbfixer/blob/master/Manual.html}{PDBFixer}
\end{itemize}

When present, we do provide rudimentary support for \texttt{pdb4amber}
via the \texttt{BioSimSpace.IO.reaadPDB} function, where passing the
\texttt{pdb4amber=True} argument will pre-process the file with
\texttt{pdb4amber} prior to creating a molecular system. However, we
choose only to support the \emph{default} options, since many are
experimental and have can have undesirable knock-on effects, e.g.~using
the \texttt{-\/-add-missing-atoms} option strips all chain identifiers
from the molecule.

\hypertarget{parameterisation}{%
\subsection{Parameterisation}\label{parameterisation}}

The
\href{https://biosimspace.org/api/index_Parameters.html}{BioSimSpace.Parameters}
package provides support for parameterising molecules using three
different engines:

\begin{itemize}
\tightlist
\item
  \href{https://ambermd.org/AmberTools.php}{AmberTools} (Using the
  \texttt{tLEaP} and \texttt{antechamber} programs.)
\item
  \href{https://manual.gromacs.org/documentation/current/onlinehelp/gmx-pdb2gmx.html}{gmx
  pdb2gmx} (Used as a fall-back for certain AMBER force fields when
  AmberTools isn't present.)
\item
  \href{https://github.com/openforcefield/openff-toolkit}{openff-toolkit}
  (The toolkit of the \href{https://openforcefield.org/}{Open Force
  Field Initiative}.)
\end{itemize}

Let's load BioSimSpace and see what force fields are available:

\begin{Shaded}
\begin{Highlighting}[]
\ImportTok{import}\NormalTok{ BioSimSpace }\ImportTok{as}\NormalTok{ BSS}

\NormalTok{BSS.Parameters.forceFields()}
\end{Highlighting}
\end{Shaded}

\begin{verbatim}
['ff03',
 'ff14SB',
 'ff99',
 'ff99SB',
 'ff99SBildn',
 'gaff',
 'gaff2',
 'openff_unconstrained-1.0.0',
 'openff_unconstrained-1.3.0',
 'openff_unconstrained-1.2.0',
 'openff_unconstrained-1.0.1',
 'openff_unconstrained-1.0.0-RC2',
 'openff_unconstrained-1.1.0',
 'openff_unconstrained-1.2.1',
 'openff_unconstrained-1.1.1',
 'openff_unconstrained-1.0.0-RC1']
\end{verbatim}

The supported force fields fall into two categories:

\begin{enumerate}
\def\labelenumi{\arabic{enumi})}
\tightlist
\item
  AMBER force fields:
\end{enumerate}

\begin{Shaded}
\begin{Highlighting}[]
\NormalTok{BSS.Parameters.amberForceFields()}
\end{Highlighting}
\end{Shaded}

\begin{verbatim}
['ff03', 'ff14SB', 'ff99', 'ff99SB', 'ff99SBildn', 'gaff', 'gaff2']
\end{verbatim}

N.B. We currently don't support force fields from \texttt{AmberTools20}
that use CMAP corrections.

\begin{enumerate}
\def\labelenumi{\arabic{enumi})}
\setcounter{enumi}{1}
\tightlist
\item
  Open Force Fields:
\end{enumerate}

\begin{Shaded}
\begin{Highlighting}[]
\NormalTok{BSS.Parameters.openForceFields()}
\end{Highlighting}
\end{Shaded}

\begin{verbatim}
['openff_unconstrained-1.0.0',
 'openff_unconstrained-1.3.0',
 'openff_unconstrained-1.2.0',
 'openff_unconstrained-1.0.1',
 'openff_unconstrained-1.0.0-RC2',
 'openff_unconstrained-1.1.0',
 'openff_unconstrained-1.2.1',
 'openff_unconstrained-1.1.1',
 'openff_unconstrained-1.0.0-RC1']
\end{verbatim}

N.B. We currently don't support the default \emph{constrained} versions
of the force fields, since we require conversion via an intermediate
\href{https://github.com/ParmEd/ParmEd}{ParmEd} topology that needs
explicit bond parameters. If required, constraints can be added at a
later stage. This will hopefully be resolved future releases when direct
translation from Open Force Field to BioSimSpace data structures should
be possible.

N.B. The available Open Force Fields are determined dynamically at
import time, so the list above might be different depending on what
version of the \texttt{openff-toolkit} you have installed.

Let's load a small molecule and parameterise it with several supported
force fields.

\begin{Shaded}
\begin{Highlighting}[]
\CommentTok{# Load a methanol molecule from a PDB file. Since there is only a single}
\CommentTok{# molecule, we take the first item from the system that was created.}
\NormalTok{molecule }\OperatorTok{=}\NormalTok{ BSS.IO.readMolecules(}\StringTok{"inputs/methanol.pdb"}\NormalTok{)[}\DecValTok{0}\NormalTok{]}
\end{Highlighting}
\end{Shaded}

As mentioned above, this is just a bare molecule that only contains
information pertaining to the topology. To see this, we can query the
\emph{properties} of the underlying
\href{https://github.com/michellab/Sire}{Sire} object.

\begin{Shaded}
\begin{Highlighting}[]
\NormalTok{molecule._sire_object.propertyKeys()}
\end{Highlighting}
\end{Shaded}

\begin{verbatim}
['insert_code',
 'beta_factor',
 'coordinates',
 'element',
 'occupancy',
 'formal_charge',
 'fileformat',
 'is_het',
 'alt_loc']
\end{verbatim}

We'll now parameterise the molecule with the
\href{http://ambermd.org/antechamber/gaff.html}{General AMBER force
field}, commonly known as GAFF. Behind the scenes this will set up and
run the
\href{http://ambermd.org/tutorials/basic/tutorial4b/}{antechamber} and
\href{https://ambermd.org/tutorials/pengfei/index.htm}{tLEaP} programs
from the \href{https://ambermd.org/AmberTools.php}{AmberTools} suite.
Depending on the input, \texttt{antechamber} might call out to
\texttt{sqm} to perform a quantum chemistry calculation in order to
calculate charges. Since this can be time consuming for a large
molecule, all of the BioSimSpace parameterisation functions return a
handle to a background process so that you can continue work
interactively while you want for the the parameterisation completes.

\begin{Shaded}
\begin{Highlighting}[]
\NormalTok{process }\OperatorTok{=}\NormalTok{ BSS.Parameters.gaff(molecule)}
\end{Highlighting}
\end{Shaded}

When you're ready to get the molecule, just call \texttt{.getMolecule()}
on the process which will block until the parameterisation is complete,
following which it will return a new molecule with force field
parameters, or raise an exception if something went wrong.

\begin{Shaded}
\begin{Highlighting}[]
\NormalTok{gaff_molecule }\OperatorTok{=}\NormalTok{ process.getMolecule()}
\end{Highlighting}
\end{Shaded}

N.B. If something went wrong, it can be useful look at the intermediate
files within \texttt{process.workDir()} to see what errors were reported
by the various programs that were run. A \texttt{README.txt} file in
this directory will also tell you exactly what commands were run, and in
what order.

Since this was just a small molecule and parameterisation was quick, we
could have just returned the molecule from the process immediately
using:

\begin{Shaded}
\begin{Highlighting}[]
\NormalTok{gaff_molecule }\OperatorTok{=}\NormalTok{ BSS.Parameters.gaff(molecule).getMolecule()}
\end{Highlighting}
\end{Shaded}

N.B. When returning immediately any intermediate files will be lost
unless the \texttt{work\_dir} parameter was used to specify a working
directory for the process.

Once again, we can query the underlying Sire object to see what
properties are associated with the molecule:

\begin{Shaded}
\begin{Highlighting}[]
\NormalTok{gaff_molecule._sire_object.propertyKeys()}
\end{Highlighting}
\end{Shaded}

\begin{verbatim}
['gb_screening',
 'coordinates',
 'charge',
 'formal_charge',
 'angle',
 'forcefield',
 'improper',
 'mass',
 'gb_radius_set',
 'element',
 'alt_loc',
 'dihedral',
 'occupancy',
 'insert_code',
 'connectivity',
 'atomtype',
 'fileformat',
 'treechain',
 'intrascale',
 'beta_factor',
 'gb_radii',
 'is_het',
 'ambertype',
 'bond',
 'LJ']
\end{verbatim}

In addition to the properties loaded from the original PDB file we now
have properties that relate to the force field parameters, such as
\texttt{bond}, \texttt{angle}, and \texttt{dihedral}.

Note that when calling \texttt{.getMolecule()} BioSimSpace copies any
additional properties from the parameterised system (created by loading
the final output from the parameterisation process) back into the a copy
of the original molecule, such that the \emph{original} topology is
\emph{preserved}. For example, while the parameterisation process may
have renamed atoms/residues, or reordered atoms, the naming and ordering
in the returned molecule will match the original that was passed in. As
mentioned earlier, we don't deal with situations where the
parameterisation engine \emph{adds} atoms that were missing from the
original topology. In this case the parameterisation would fail, since
the new topology is inconsistent with the original.

Let us now parameterise the same molecule using one of the Open Force
Fields:

\begin{Shaded}
\begin{Highlighting}[]
\NormalTok{openff_molecule }\OperatorTok{=}\NormalTok{ BSS.Parameters.openff_unconstrained_1_0_0(molecule).getMolecule()}
\end{Highlighting}
\end{Shaded}

We can now loop over atoms in the two parameterised molecules and
compare properties. For example, we can see that the atomic charges are
the same:

\begin{Shaded}
\begin{Highlighting}[]
\ControlFlowTok{for}\NormalTok{ atom0, atom1 }\KeywordTok{in} \BuiltInTok{zip}\NormalTok{(gaff_molecule.getAtoms(), openff_molecule.getAtoms()):}
    \BuiltInTok{print}\NormalTok{(atom0.name(), atom0.charge(), atom1.charge())}
\end{Highlighting}
\end{Shaded}

\begin{verbatim}
C 0.1167 |e| 0.1167 |e|
H1 0.0287 |e| 0.0287 |e|
H2 0.0287 |e| 0.0287 |e|
H3 0.0287 |e| 0.0287 |e|
OH -0.5988 |e| -0.5988 |e|
HO 0.3960 |e| 0.3960 |e|
\end{verbatim}

To compare specific terms in the force field we can query the properties
of the underlying Sire objects:

\begin{Shaded}
\begin{Highlighting}[]
\CommentTok{# Get the bond potentials generated by GAFF.}
\NormalTok{gaff_molecule._sire_object.}\BuiltInTok{property}\NormalTok{(}\StringTok{"bond"}\NormalTok{).potentials()}
\end{Highlighting}
\end{Shaded}

\begin{verbatim}
[TwoAtomFunction( {CGIdx(0),Index(0)} <-> {CGIdx(0),Index(2)} : 330.6 [r - 1.0969]^2 ),
 TwoAtomFunction( {CGIdx(0),Index(0)} <-> {CGIdx(0),Index(3)} : 330.6 [r - 1.0969]^2 ),
 TwoAtomFunction( {CGIdx(0),Index(0)} <-> {CGIdx(0),Index(1)} : 330.6 [r - 1.0969]^2 ),
 TwoAtomFunction( {CGIdx(0),Index(4)} <-> {CGIdx(0),Index(5)} : 371.4 [r - 0.973]^2 ),
 TwoAtomFunction( {CGIdx(0),Index(0)} <-> {CGIdx(0),Index(4)} : 316.7 [r - 1.4233]^2 )]
\end{verbatim}

\begin{Shaded}
\begin{Highlighting}[]
\CommentTok{# Get the bond potentials generated by OpenFF.}
\NormalTok{openff_molecule._sire_object.}\BuiltInTok{property}\NormalTok{(}\StringTok{"bond"}\NormalTok{).potentials()}
\end{Highlighting}
\end{Shaded}

\begin{verbatim}
[TwoAtomFunction( {CGIdx(0),Index(0)} <-> {CGIdx(0),Index(2)} : 379.047 [r - 1.09289]^2 ),
 TwoAtomFunction( {CGIdx(0),Index(0)} <-> {CGIdx(0),Index(3)} : 379.047 [r - 1.09289]^2 ),
 TwoAtomFunction( {CGIdx(0),Index(0)} <-> {CGIdx(0),Index(1)} : 379.047 [r - 1.09289]^2 ),
 TwoAtomFunction( {CGIdx(0),Index(4)} <-> {CGIdx(0),Index(5)} : 560.292 [r - 0.970769]^2 ),
 TwoAtomFunction( {CGIdx(0),Index(0)} <-> {CGIdx(0),Index(4)} : 334.571 [r - 1.41429]^2 )]
\end{verbatim}

As well as being able to parameterise a molecule loaded from file,
BioSimSpace can also use a
\href{https://en.wikipedia.org/wiki/Simplified_molecular-input_line-entry_system}{SMILES}
string as the molecule parameter for any force field function. This can
be useful if you want to highlight particular steroechemistry, which
might not be possible in the intermediate file formats that are used
behind the scenes duing the parameterisation, e.g.~PDB. When using
SMILES there is no constraint that the parameterised topology matches
that of the original molecule, since we have not yet created a
fully-fledged BioSimSpace molecule at the point at which we invoke the
parameterisation. As such, it is perfectly acceptable for the
parameterisation to add hydrogen atoms.

With this in mind, the two parameterisations shown above could also have
been performed as follows:

\begin{Shaded}
\begin{Highlighting}[]
\NormalTok{gaff_molecule }\OperatorTok{=}\NormalTok{ BSS.Parameters.gaff(}\StringTok{"CO"}\NormalTok{).getMolecule()}
\NormalTok{openff_molecule }\OperatorTok{=}\NormalTok{ BSS.Parameters.openff_unconstrained_1_0_0(}\StringTok{"CO"}\NormalTok{).getMolecule()}
\end{Highlighting}
\end{Shaded}

Here hydrogen atoms have been added during the parameterisation. While
the atom layout and naming is different those of the previous examples,
the parameters for the equivalent atoms are the same, e.g.:

\begin{Shaded}
\begin{Highlighting}[]
\ControlFlowTok{for}\NormalTok{ atom0, atom1 }\KeywordTok{in} \BuiltInTok{zip}\NormalTok{(gaff_molecule.getAtoms(), openff_molecule.getAtoms()):}
    \BuiltInTok{print}\NormalTok{(atom0.name(), atom0.charge(), atom1.charge())}
\end{Highlighting}
\end{Shaded}

\begin{verbatim}
C1 0.1167 |e| 0.1167 |e|
O1 -0.5988 |e| -0.5988 |e|
H1 0.0287 |e| 0.0287 |e|
H2 0.0287 |e| 0.0287 |e|
H3 0.0287 |e| 0.0287 |e|
H4 0.3960 |e| 0.3960 |e|
\end{verbatim}

In addition to small molecules, BioSimSpace provides support for
parameterising proteins using force fields from \texttt{AmberTools},
such as
\href{https://pubs.acs.org/doi/abs/10.1021/acs.jctc.5b00255}{ff14SB}.
(We don't currently support
\href{https://pubs.acs.org/doi/10.1021/acs.jctc.9b00591}{ff19SB} due to
the presence of CMAP terms, which we don't yet support in our parsers.)

As an example:

\begin{Shaded}
\begin{Highlighting}[]
\NormalTok{protein }\OperatorTok{=}\NormalTok{ BSS.IO.readMolecules(}\StringTok{"inputs/2JJC.pdb"}\NormalTok{)[}\DecValTok{0}\NormalTok{]}
\NormalTok{protein }\OperatorTok{=}\NormalTok{ BSS.Parameters.ff14SB(protein).getMolecule()}
\end{Highlighting}
\end{Shaded}

When molecules contain bound ions it is necessary to choose a water
model for the ion parameters. This can be achieved by passing the
\texttt{water\_model} argument to any parameterisation function, where
the named water model must match one of those described in the
\textbf{Solvation} section below, e.g. \texttt{water\_model="tip3p"}.
(An exception will be raised if bound ions are detected and no water
model is chosen.) An optional \texttt{leap\_commands} argument allows
you to pass additional directives to the \texttt{tLEaP} program called
by an AMBER protein force field function. These commands are added after
the defaults allowing you to load custom force field parameters, etc.

\hypertarget{solvation}{%
\subsection{Solvation}\label{solvation}}

The next stage in setting up a system ready for simulation is to solvate
the molecule(s) in a box of water. The
\href{https://biosimspace.org/api/index_Solvent.html}{BioSimSpace.Solvent}
package provides support for solvating with a variety of water models:

\begin{Shaded}
\begin{Highlighting}[]
\NormalTok{BSS.Solvent.waterModels()}
\end{Highlighting}
\end{Shaded}

\begin{verbatim}
['spc', 'spce', 'tip3p', 'tip4p', 'tip5p']
\end{verbatim}

N.B. At present we only support solvating in water.

Solvation is performed using the
\href{https://manual.gromacs.org/documentation/2018/onlinehelp/gmx-solvate.html}{gmx
solvate} package. We currently don't support other solvation engines,
since they require parameterisation as a pre-requisite, or include
parameterisation as part of the solvation process itself, i.e.~you can't
decouple the two stages. We will hopefully overcome this shortcoming in
future releases, since other engines would enable support for
\href{https://www.ncbi.nlm.nih.gov/pmc/articles/PMC6078207/}{more
realistic salt concentrations} and improved water templates, i.e.~less
vapour bubbles at box edges, which need to be properly equlibrated.

BioSimSpace provides support for both orthorhombic and triclinic
simulation boxes, where appropriate box magnitudes and angles can be
obtained using the
\href{https://biosimspace.org/api/index_Box.html}{BioSimSpace.Box}
package. To see what pre-generated box types are available:

\begin{Shaded}
\begin{Highlighting}[]
\NormalTok{BSS.Box.boxTypes()}
\end{Highlighting}
\end{Shaded}

\begin{verbatim}
['cubic',
 'rhombicDodecahedronHexagon',
 'rhombicDodecahedronSquare',
 'truncatedOctahedron']
\end{verbatim}

For example, to get box parameters for a truncated octahedral box with
an image distance of 10 nanometers.

\begin{Shaded}
\begin{Highlighting}[]
\NormalTok{box, angles }\OperatorTok{=}\NormalTok{ BSS.Box.truncatedOctahedron(}\DecValTok{10}\OperatorTok{*}\NormalTok{BSS.Units.Length.nanometer)}
\BuiltInTok{print}\NormalTok{(box, angles)}
\end{Highlighting}
\end{Shaded}

\begin{verbatim}
[100.0000 A, 100.0000 A, 100.0000 A] [109.4712 degrees, 70.5288 degrees, 70.5288 degrees]
\end{verbatim}

When choosing a box for solvation it is important to ensure that it
large enough to hold the molecule(s) of interest. In addition, when
adding ions to neutralise the system or reach a desired ionic strength,
then the system must be large enough that the cut-off used by
\href{https://manual.gromacs.org/archive/5.0/programs/gmx-genion.html}{gmx
genion} when computing electrostatics is not more than twice the
shortest box dimension. This can lead to a slight self-consistency
issue, since it's sometimes not possible to know the number of ions that
will need to be added a priori. Even if that were possible, then the
random insertion of ions by \texttt{gmx\ genion} can still lead to
issues if the choice of location means that they don't all manage to fit
within the available volume.

As such, sometimes a little trial-and-error is needed to find an
appropriate box size for the system in question. A good rule of thumb is
to obtain the
\href{https://en.wikipedia.org/wiki/Bounding_volume}{axis-aligned
bounding box} for the molecule(s) and add an appropriate buffer to the
largest box dimension. For example:

\begin{Shaded}
\begin{Highlighting}[]
\CommentTok{# Get the minimium and maximum coordinates of the bounding box that}
\CommentTok{# minimally encloses the protein.}
\NormalTok{box_min, box_max }\OperatorTok{=}\NormalTok{ protein.getAxisAlignedBoundingBox()}

\CommentTok{# Work out the box size from the difference in the coordinates.}
\NormalTok{box_size }\OperatorTok{=}\NormalTok{ [y }\OperatorTok{-}\NormalTok{ x }\ControlFlowTok{for}\NormalTok{ x, y }\KeywordTok{in} \BuiltInTok{zip}\NormalTok{(box_min, box_max)]}

\CommentTok{# How much to pad each side of the protein? (Nonbonded cutoff = 10 A)}
\NormalTok{padding }\OperatorTok{=} \DecValTok{15} \OperatorTok{*}\NormalTok{ BSS.Units.Length.angstrom}

\CommentTok{# Work out an appropriate box. This will used in each dimension to ensure}
\CommentTok{# that the cutoff constraints are satisfied if the molecule rotates.}
\NormalTok{box_length }\OperatorTok{=} \BuiltInTok{max}\NormalTok{(box_size) }\OperatorTok{+} \DecValTok{2}\OperatorTok{*}\NormalTok{padding}
\end{Highlighting}
\end{Shaded}

Armed with this information, we can now solvate our protein in an
appropriately sized cubic box. Here we will use the TIP3P water model:

\begin{Shaded}
\begin{Highlighting}[]
\NormalTok{solvated }\OperatorTok{=}\NormalTok{ BSS.Solvent.tip3p(molecule}\OperatorTok{=}\NormalTok{protein, box}\OperatorTok{=}\DecValTok{3}\OperatorTok{*}\NormalTok{[box_length])}
\end{Highlighting}
\end{Shaded}

N.B. The \texttt{molecule} argument is optional. If ommited, then a pure
water box will be generated.

By default, BioSimSpace will add counter-ions to neutralise the system.
To see what ions were added we can use the built in \texttt{search}
functionality:

\begin{Shaded}
\begin{Highlighting}[]
\CommentTok{# Search for all free ions. As a simple search, we look for all molecules}
\CommentTok{# that only contain a single atom.}
\NormalTok{search }\OperatorTok{=}\NormalTok{ solvated.search(}\StringTok{"not mols with atomidx 2"}\NormalTok{)}

\CommentTok{# Print all ions and their charge.}
\ControlFlowTok{for}\NormalTok{ ion }\KeywordTok{in}\NormalTok{ search:}
    \BuiltInTok{print}\NormalTok{(}\SpecialStringTok{f"element = }\SpecialCharTok{\{}\NormalTok{ion}\SpecialCharTok{.}\NormalTok{element()}\SpecialCharTok{\}}\SpecialStringTok{, charge = }\SpecialCharTok{\{}\NormalTok{ion}\SpecialCharTok{.}\NormalTok{charge()}\SpecialCharTok{\}}\SpecialStringTok{"}\NormalTok{)}
\end{Highlighting}
\end{Shaded}

\begin{verbatim}
element = Sodium (Na, 11), charge = 1.0000 |e|
element = Sodium (Na, 11), charge = 1.0000 |e|
element = Sodium (Na, 11), charge = 1.0000 |e|
element = Sodium (Na, 11), charge = 1.0000 |e|
element = Sodium (Na, 11), charge = 1.0000 |e|
element = Sodium (Na, 11), charge = 1.0000 |e|
element = Sodium (Na, 11), charge = 1.0000 |e|
\end{verbatim}

He're we see that \texttt{gmx\ genion} added 7 sodium ions. To confirm
that the system was indeed neutralised, we can check its charge, as well
as the charge of the original protein:

\begin{Shaded}
\begin{Highlighting}[]
\BuiltInTok{print}\NormalTok{(}\SpecialStringTok{f"solvated = }\SpecialCharTok{\{}\NormalTok{solvated}\SpecialCharTok{.}\NormalTok{charge()}\SpecialCharTok{\}}\SpecialStringTok{, protein = }\SpecialCharTok{\{}\NormalTok{protein}\SpecialCharTok{.}\NormalTok{charge()}\SpecialCharTok{\}}\SpecialStringTok{"}\NormalTok{)}
\end{Highlighting}
\end{Shaded}

\begin{verbatim}
solvated = -1.2183e-07 |e|, protein = -7.0000 |e|
\end{verbatim}

We now have a solvated system ready for simulation. Let's visualise it
with
\href{https://biosimspace.org/api/generated/BioSimSpace.Notebook.View.html\#BioSimSpace.Notebook.View}{BioSimSpace.Notebook.View}:

\begin{Shaded}
\begin{Highlighting}[]
\NormalTok{view }\OperatorTok{=}\NormalTok{ BSS.Notebook.View(solvated)}
\NormalTok{view.system()}
\end{Highlighting}
\end{Shaded}

\begin{figure}
\centering
\includegraphics{https://github.com/michellab/BioSimSpaceTutorials/blob/d26d5bcfe8e66fa6c9aa8bdb9788fb5eeb252ee8/01_introduction/assets/02_solvated.png}
\caption{Visualisation of the solvated system.}
\end{figure}

Finally, let's save the system to file in AMBER format.

\begin{Shaded}
\begin{Highlighting}[]
\NormalTok{BSS.IO.saveMolecules(}\StringTok{"solvated"}\NormalTok{, solvated, [}\StringTok{"prm7"}\NormalTok{, }\StringTok{"rst7"}\NormalTok{])}
\end{Highlighting}
\end{Shaded}

\begin{verbatim}
['/home/lester/Code/BioSimSpaceTutorials/01_introduction/solvated.prm7',
 '/home/lester/Code/BioSimSpaceTutorials/01_introduction/solvated.rst7']
\end{verbatim}

Author: Lester Hedges Email:~~ lester.hedges@bristol.ac.uk

\hypertarget{molecular-dynamics}{%
\section{Molecular dynamics}\label{molecular-dynamics}}

The companion notebook for this section can be found
\href{https://github.com/michellab/BioSimSpaceTutorials/blob/4844562e7d2cd0b269cead56562ec16a3dfaef7c/01_introduction/03_molecular_dynamics.ipynb}{here}

\hypertarget{introduction}{%
\subsection{Introduction}\label{introduction}}

In this section we will learn how to use BioSimSpace to configure and
run some basic molecular dynamics simulations.

\hypertarget{protocols}{%
\subsection{Protocols}\label{protocols}}

One of the key goals of BioSimSpace was to start a conversation
regarding \emph{best practice} within the biomolecular simulation
community and to facilitate the codifying of shareable, re-usable, and
extensible simulation protocols.

The
\href{https://biosimspace.org/api/index_Protocol.html}{BioSimSpace.Protocol}
package defines protocols for a range of common molecular dynamics
simulations. We can query the package to ee what protocols are
available:

\begin{Shaded}
\begin{Highlighting}[]
\ImportTok{import}\NormalTok{ BioSimSpace }\ImportTok{as}\NormalTok{ BSS}
\NormalTok{BSS.Protocol.protocols()}
\end{Highlighting}
\end{Shaded}

\begin{verbatim}
['Equilibration',
 'FreeEnergy',
 'Metadynamics',
 'Minimisation',
 'Production',
 'Steering']
\end{verbatim}

Since we require protocols to be \emph{interoperable}, the classes
listed above are simple objects that allow you to configure a
\emph{limited} set of options that are handled by \emph{all} of the
molecular dynamics engines that we support. This might seem quite
restrictive, but we will see later how it is possible to fully customise
a simulation for a particular molecular dynamics engine.

Each protocol comes with some default options. To see what those are we
can instantiate an object using the default constructor. For example,
let's explore the
\href{https://biosimspace.org/api/generated/BioSimSpace.Protocol.Equilibration.html\#BioSimSpace.Protocol.Equilibration}{Equilibration}
protocol.

\begin{Shaded}
\begin{Highlighting}[]
\NormalTok{protocol }\OperatorTok{=}\NormalTok{ BSS.Protocol.Equilibration()}
\BuiltInTok{print}\NormalTok{(protocol)}
\end{Highlighting}
\end{Shaded}

\begin{verbatim}
<BioSimSpace.Protocol.Equilibration: timestep=2.0000 fs, runtime=0.2000 ns, temperature_start=300.0000 K, temperature_end=300.0000 K, pressure=None, report_interval=100, restart_interval=500,restraint=None>
\end{verbatim}

Here we can see that the default protocol performs an equlibration at
fixed temperature (\texttt{temperature\_start\ ==\ temperature\_end}) in
the NVT ensemble (\texttt{pressure=None}) with no restraints
(\texttt{restraint=None}). The total simulation time is 0.2 nanoseconds
with an integration timestep of 2 femtoseconds. The
\texttt{report\_interval} and \texttt{restart\_interval} govern how
frequently information is written to log and restart (and/or trajectory)
files respectively.

If any of these defaults are unsuitable, then you are free to change
them by passing in appropriate values for each of the arguments when
instantiating the object. In some cases it might be desirable to
\emph{override} the default protocols and set specific values of the
arguments that are suitable for a particular project or team. This can
be achieved by defining a set of function wrappers that configure and
return the protocols using your own defaults.

As an example, the following configuration could be used to provide
alternative defaults for NVT and NPT equlibration protocols. (You could
simply re-use the existing protocol name, but here we provide two
separate protocols for convenience.)

\begin{Shaded}
\begin{Highlighting}[]
\CommentTok{# myconfig/Protocol.py}
\ImportTok{import}\NormalTok{ BioSimSpace }\ImportTok{as}\NormalTok{ BSS}

\CommentTok{# Override the equilibration protocol with some custom defaults. Ideally all}
\CommentTok{# arguments to the BioSimSpace function would be mapped, but here we use a}
\CommentTok{# subset for simplicity.}

\CommentTok{# A custom equlibration in the NVT ensemble.}
\KeywordTok{def}\NormalTok{ EquilibrationNVT(runtime}\OperatorTok{=}\DecValTok{5}\OperatorTok{*}\NormalTok{BSS.Units.Time.nanosecond,}
\NormalTok{                     report_interval}\OperatorTok{=}\DecValTok{2500}\NormalTok{,}
\NormalTok{                     restart_interval}\OperatorTok{=}\DecValTok{250000}\NormalTok{,}
\NormalTok{                     restraint}\OperatorTok{=}\StringTok{"backbone"}\NormalTok{):}
    \ControlFlowTok{return}\NormalTok{ BSS.Protocol.Equilibration(runtime}\OperatorTok{=}\NormalTok{runtime,}
\NormalTok{                                      report_interval}\OperatorTok{=}\NormalTok{report_interval,}
\NormalTok{                                      restart_interval}\OperatorTok{=}\NormalTok{restart_interval,}
\NormalTok{                                      restraint}\OperatorTok{=}\NormalTok{restraint)}

\CommentTok{# A custom equlibration in the NPT ensemble.}
\KeywordTok{def}\NormalTok{ EquilibrationNPT(runtime}\OperatorTok{=}\DecValTok{5}\OperatorTok{*}\NormalTok{BSS.Units.Time.nanosecond,}
\NormalTok{                     pressure}\OperatorTok{=}\NormalTok{BSS.Units.Pressure.atm,}
\NormalTok{                     report_interval}\OperatorTok{=}\DecValTok{2500}\NormalTok{,}
\NormalTok{                     restart_interval}\OperatorTok{=}\DecValTok{25000}\NormalTok{,}
\NormalTok{                     restraint}\OperatorTok{=}\StringTok{"backbone"}\NormalTok{):}
    \ControlFlowTok{return}\NormalTok{ BSS.Protocol.Equilibration(runtime}\OperatorTok{=}\NormalTok{runtime,}
\NormalTok{                                      pressure}\OperatorTok{=}\NormalTok{pressure,}
\NormalTok{                                      report_interval}\OperatorTok{=}\NormalTok{report_interval,}
\NormalTok{                                      restart_interval}\OperatorTok{=}\NormalTok{restart_interval,}
\NormalTok{                                      restraint}\OperatorTok{=}\NormalTok{restraint)}
\end{Highlighting}
\end{Shaded}

We could then import the customised protocols from our local
configuraton and use them instead, e.g.:

\begin{Shaded}
\begin{Highlighting}[]
\ImportTok{from}\NormalTok{ myconfig.Protocol }\ImportTok{import} \OperatorTok{*}

\NormalTok{protocol }\OperatorTok{=}\NormalTok{ EquilibrationNVT()}
\BuiltInTok{print}\NormalTok{(protocol)}
\end{Highlighting}
\end{Shaded}

\begin{verbatim}
<BioSimSpace.Protocol.Equilibration: timestep=2.0000 fs, runtime=5.0000 ns, temperature_start=300.0000 K, temperature_end=300.0000 K, pressure=None, report_interval=2500, restart_interval=250000,restraint='backbone'>
\end{verbatim}

\hypertarget{processes}{%
\subsection{Processes}\label{processes}}

Once you have created a molecular system and chosen a protocol, then it
is time to create a simulation \emph{process}. The
\href{https://biosimspace.org/api/index_Process.html}{BioSimSpace.Process}
package provides functionality for configuring and running processes
with several common molecular dynamics engines.

Let's query the package to see what engines are available:

\begin{Shaded}
\begin{Highlighting}[]
\NormalTok{BSS.Process.engines()}
\end{Highlighting}
\end{Shaded}

\begin{verbatim}
['Amber', 'Gromacs', 'Namd', 'OpenMM', 'Somd']
\end{verbatim}

Before creating a process let us once again load our example
alanine-dipeptide system from file:

\begin{Shaded}
\begin{Highlighting}[]
\NormalTok{system }\OperatorTok{=}\NormalTok{ BSS.IO.readMolecules(}\StringTok{"inputs/ala*"}\NormalTok{)}
\end{Highlighting}
\end{Shaded}

As a simple example, let us use a short minimisation protocol:

\begin{Shaded}
\begin{Highlighting}[]
\NormalTok{protocol }\OperatorTok{=}\NormalTok{ BSS.Protocol.Minimisation(steps}\OperatorTok{=}\DecValTok{1000}\NormalTok{)}
\end{Highlighting}
\end{Shaded}

We'll now create a process to apply the \texttt{Protocol} to the
\texttt{System} using the AMBER molecular dynamics engine:

\begin{Shaded}
\begin{Highlighting}[]
\NormalTok{process }\OperatorTok{=}\NormalTok{ BSS.Process.Amber(system, protocol)}
\end{Highlighting}
\end{Shaded}

A lot of complexity is hidden in this line. BioSimSpace has
automatically found an AMBER executable on the underlying operating
system, has automatically written AMBER format molecular input files,
generated an AMBER configuration file for the minimisation protocol, and
configured any command-line arguments that are required.

By default, processes are run inside of a temporary working directory
hidden away from the user. To see where this is, run:

\begin{Shaded}
\begin{Highlighting}[]
\NormalTok{process.workDir()}
\end{Highlighting}
\end{Shaded}

\begin{verbatim}
'/tmp/tmpjuzioj_o'
\end{verbatim}

N.B. If you want to use a different temporary directory, e.g.~one with a
faster disk, then simply set the \texttt{TMPDIR} environment variable.
Alternatively, you can pass the \texttt{work\_dir} argument to the
\texttt{Process} constructor to explicitly specify the path. This can be
useful when you want named directories, or want to examine the
intermediate files from the \texttt{Process} for debugging purposes.

To see what executable was found, run:

\begin{Shaded}
\begin{Highlighting}[]
\NormalTok{process.exe()}
\end{Highlighting}
\end{Shaded}

\begin{verbatim}
'/home/lester/sire.app/bin/sander'
\end{verbatim}

To see the list of autogenerated input files:

\begin{Shaded}
\begin{Highlighting}[]
\NormalTok{process.inputFiles()}
\end{Highlighting}
\end{Shaded}

\begin{verbatim}
['/tmp/tmpjuzioj_o/amber.cfg',
 '/tmp/tmpjuzioj_o/amber.rst7',
 '/tmp/tmpjuzioj_o/amber.prm7']
\end{verbatim}

If you like, we could zip up the input files to use on another occasion.
When working on a notebook server it's possible to return a file link so
that we can download them:

\begin{Shaded}
\begin{Highlighting}[]
\NormalTok{process.getInput(file_link}\OperatorTok{=}\VariableTok{True}\NormalTok{)}
\end{Highlighting}
\end{Shaded}

amber\_input.zip

We can query also query the list of configuration file options:

\begin{Shaded}
\begin{Highlighting}[]
\NormalTok{process.getConfig()}
\end{Highlighting}
\end{Shaded}

\begin{verbatim}
['Minimisation',
 ' &cntrl',
 '  imin=1,',
 '  ntx=1,',
 '  ntxo=1,',
 '  ntpr=100,',
 '  irest=0,',
 '  maxcyc=1000,',
 '  ncyc=1000,',
 '  cut=8.0,',
 ' /']
\end{verbatim}

And also get command-line argument string for the process:

\begin{Shaded}
\begin{Highlighting}[]
\NormalTok{process.getArgString()}
\end{Highlighting}
\end{Shaded}

\begin{verbatim}
'-O -i amber.cfg -p amber.prm7 -c amber.rst7 -o stdout -r amber.crd -inf amber.nrg'
\end{verbatim}

If you're an expert in a particular package then BioSimSpace allows you
to fully customise the process by tweaking the configuration options and
command-line arguments. Read the help documentation for
\texttt{process.setConfig} and \texttt{process.setArgs} if you are
interested. Once again, it's possible to wrap the instantiation of
\texttt{Process} objects in your own custom functions, allowing you to
tweak the default configuration options for your own requirements. For
example, if you always want to wrap coordinates to the minimum image
when using AMBER, then this could be achieved as follows:

\begin{Shaded}
\begin{Highlighting}[]
\CommentTok{# myconfig/Process.py}
\ImportTok{import}\NormalTok{ BioSimSpace }\ImportTok{as}\NormalTok{ BSS}

\CommentTok{# Wrap the instantiation of BSS.Process.Amber objects to configure them}
\CommentTok{# such that coordinates are always wrapped to the minimum image.}
\KeywordTok{def}\NormalTok{ Amber(system, protocol, exe}\OperatorTok{=}\VariableTok{None}\NormalTok{, name}\OperatorTok{=}\StringTok{"amber"}\NormalTok{,}
\NormalTok{            work_dir}\OperatorTok{=}\VariableTok{None}\NormalTok{, seed}\OperatorTok{=}\VariableTok{None}\NormalTok{, property_map}\OperatorTok{=}\NormalTok{\{\}):}

    \CommentTok{# Create process using the passed parameters.}
\NormalTok{    process }\OperatorTok{=}\NormalTok{ BSS.Process.Amber(system,}
\NormalTok{                                protocol,}
\NormalTok{                                exe}\OperatorTok{=}\NormalTok{exe,}
\NormalTok{                                name}\OperatorTok{=}\NormalTok{name,}
\NormalTok{                                work_dir}\OperatorTok{=}\NormalTok{work_dir,}
\NormalTok{                                seed}\OperatorTok{=}\NormalTok{seed,}
\NormalTok{                                property_map}\OperatorTok{=}\NormalTok{property_map)}
    
    \CommentTok{# Get the config.}
\NormalTok{    config }\OperatorTok{=}\NormalTok{ process.getConfig()}
    
    \CommentTok{# Add coordinate wrapping to the end of the config.}
\NormalTok{    config[}\OperatorTok{-}\DecValTok{1}\NormalTok{] }\OperatorTok{=} \StringTok{"  iwrap=1,"}
\NormalTok{    config.append(}\StringTok{" /"}\NormalTok{)}

    \CommentTok{# Set the new config.}
\NormalTok{    process.setConfig(config)}
    
    \CommentTok{# Return the process.}
    \ControlFlowTok{return}\NormalTok{ process}
\end{Highlighting}
\end{Shaded}

Let us know create our custom AMBER process and check the configuration:

\begin{Shaded}
\begin{Highlighting}[]
\ImportTok{from}\NormalTok{ myconfig.Process }\ImportTok{import} \OperatorTok{*}

\NormalTok{process }\OperatorTok{=}\NormalTok{ Amber(system, protocol)}
\NormalTok{process.getConfig()}
\end{Highlighting}
\end{Shaded}

\begin{verbatim}
['Minimisation',
 ' &cntrl',
 '  imin=1,',
 '  ntx=1,',
 '  ntxo=1,',
 '  ntpr=100,',
 '  irest=0,',
 '  maxcyc=1000,',
 '  ncyc=1000,',
 '  cut=8.0,',
 '  iwrap=1,',
 ' /']
\end{verbatim}

N.B. You might want to add additional configuration details to your
\texttt{Process} wrappers, e.g.~to ensure that a specific executable is
used.

Now that we have a process, let's go ahead and start it:

\begin{Shaded}
\begin{Highlighting}[]
\NormalTok{process.start()}
\end{Highlighting}
\end{Shaded}

\begin{verbatim}
BioSimSpace.Process.Amber(<BioSimSpace.System: nMolecules=631>, <BioSimSpace.Protocol.Custom>, exe='/home/lester/sire.app/bin/sander', name='amber', work_dir='/tmp/tmpm8kx_jms', seed=None)
\end{verbatim}

BioSimSpace has now launched a minimisation process in the background!
When in an interactive session you carry on working and periodically
check in on the process to see how its doing.

To check whether the process is running:

\begin{Shaded}
\begin{Highlighting}[]
\NormalTok{process.isRunning()}
\end{Highlighting}
\end{Shaded}

\begin{verbatim}
True
\end{verbatim}

We can see how many minutes it has been running for:

\begin{Shaded}
\begin{Highlighting}[]
\NormalTok{process.runTime()}
\end{Highlighting}
\end{Shaded}

\begin{verbatim}
0.1960 mins
\end{verbatim}

Since this is a short minimisation it will likely finish pretty quickly.
Let's print the final energy of the system and return the minimised
molecular configuration.

\begin{Shaded}
\begin{Highlighting}[]
\BuiltInTok{print}\NormalTok{(process.getTotalEnergy(block}\OperatorTok{=}\VariableTok{True}\NormalTok{))}
\NormalTok{minimised }\OperatorTok{=}\NormalTok{ process.getSystem()}
\end{Highlighting}
\end{Shaded}

\begin{verbatim}
-6954.7000 kcal/mol
\end{verbatim}

When working interactively, any time we query a running process we get
back the \emph{latest} information that has been written to disk. This
means that we can get an update on how things are progressing, then
immediately carry on with what we were doing in our notebook. By passing
\texttt{block=True}, as we do when we call \texttt{getTotalEnergy}
above, we request that the process finishes running before returning a
result. This means we get the \emph{final} energy, and the minimised
system that is returned afterwards represents the \emph{final} snapshot
that was saved.

Let's now re-run the simulation, instead using GROMACS as the MD engine.

\begin{Shaded}
\begin{Highlighting}[]
\NormalTok{process }\OperatorTok{=}\NormalTok{ BSS.Process.Gromacs(system, protocol)}
\end{Highlighting}
\end{Shaded}

When the process is instantiated, BioSimSpace takes the system that was
read from AMBER format files and converts it to GROMACS format ready for
simulation. Let's take a look at the list of input files that were
autogenerated for us:

\begin{Shaded}
\begin{Highlighting}[]
\NormalTok{process.inputFiles()}
\end{Highlighting}
\end{Shaded}

\begin{verbatim}
['/tmp/tmpsxf5hoej/gromacs.mdp',
 '/tmp/tmpsxf5hoej/gromacs.gro',
 '/tmp/tmpsxf5hoej/gromacs.top',
 '/tmp/tmpsxf5hoej/gromacs.tpr']
\end{verbatim}

Let's start the process running and, once again, wait for it to finish
before getting the minimised system.

\begin{Shaded}
\begin{Highlighting}[]
\NormalTok{process.start()}
\NormalTok{minimised }\OperatorTok{=}\NormalTok{ process.getSystem(block}\OperatorTok{=}\VariableTok{True}\NormalTok{)}
\end{Highlighting}
\end{Shaded}

\hypertarget{interactive-molecular-dynamics}{%
\subsection{Interactive molecular
dynamics}\label{interactive-molecular-dynamics}}

The example in the previous section was finished almost as soon as it
began. Let's run a more complicated equilibration protocol so that we
can learn more about how to monitor processes interactively using
BioSimSpace.

\begin{Shaded}
\begin{Highlighting}[]
\NormalTok{protocol }\OperatorTok{=}\NormalTok{ BSS.Protocol.Equilibration(runtime}\OperatorTok{=}\DecValTok{20}\OperatorTok{*}\NormalTok{BSS.Units.Time.picosecond,}
\NormalTok{                                      temperature_start}\OperatorTok{=}\DecValTok{0}\OperatorTok{*}\NormalTok{BSS.Units.Temperature.kelvin,}
\NormalTok{                                      temperature_end}\OperatorTok{=}\DecValTok{300}\OperatorTok{*}\NormalTok{BSS.Units.Temperature.kelvin,}
\NormalTok{                                      restraint}\OperatorTok{=}\StringTok{"backbone"}\NormalTok{)}
\end{Highlighting}
\end{Shaded}

This protocol will equlibrate a system for 20 picoseconds, while heating
it from 0 to 300 Kelvin and restraining any atoms in the backbone of the
molecule. Note that some of the parameters passed have units, e.g.~the
temperatures are in Kelvin. BioSimSpace has a built in type system for
handling variables with units. The \texttt{BSS.Units} package provides a
convenient way of declaring these, for example
\texttt{10*BSS.Units.Temperature.kelvin} creates an object of type
\texttt{BSS.Types.Temperature} with a magnitude of 10 and unit of
Kelvin. This allows the user to pass parameters with whatever unit they
like. BioSimSpace will simply convert it to the correct unit for the
chosen MD engine internally.

One again, we now need a \texttt{Process} in order to run our
simulation. Exectute the cell below to initialise an AMBER process and
start it immediately. Note that we pass in the minimised system from the
last example, along with our new protocol.

\begin{Shaded}
\begin{Highlighting}[]
\NormalTok{process }\OperatorTok{=}\NormalTok{ BSS.Process.Amber(minimised, protocol).start()}
\end{Highlighting}
\end{Shaded}

We can monitor the time, temperature, and energy as the process runs. If
you run this multiple times using ``CTRL+Return'' you'll see the
temperature slowly increasing.

\begin{Shaded}
\begin{Highlighting}[]
\BuiltInTok{print}\NormalTok{(process.getTime(), process.getTemperature(), process.getTotalEnergy())}
\end{Highlighting}
\end{Shaded}

\begin{verbatim}
2.4000 ps 30.6900 K -6936.6583 kcal/mol
\end{verbatim}

Since all of the values returned above are typed we can easily convert
them to other units:

\begin{Shaded}
\begin{Highlighting}[]
\BuiltInTok{print}\NormalTok{(process.getTime().nanoseconds(), process.getTemperature().celsius(), process.getTotalEnergy().kj_per_mol())}
\end{Highlighting}
\end{Shaded}

\begin{verbatim}
0.0050 ns -204.7600 C -2.7995e+04 kJ/mol
\end{verbatim}

It's possible to query many other thermodynamic records. What's
available depends on type of protocol and the MD package that is used to
run the protocol. To get more information, run:

N.B. Certain functionality is specific to the process in question, i.e.
\texttt{BSS.Process.Amber} will have different options to
\texttt{BSS.Process.Gromacs}, but, for the purposes of interoperability,
there is a core set of functionality that is consistent across all
\texttt{Process} classes, e.g.~all classes implement a
\texttt{getSystem} method.)

\hypertarget{plotting-time-series-data}{%
\subsubsection{Plotting time series
data}\label{plotting-time-series-data}}

As well as querying the most recent records we can also get a time
series of results by passing the \texttt{time\_series} keyword argument
to any of the data record getter methods, e.g.

\begin{Shaded}
\begin{Highlighting}[]
\CommentTok{# Get a time series of pressure records.}
\NormalTok{pressure }\OperatorTok{=}\NormalTok{ process.getPressure(time_series}\OperatorTok{=}\VariableTok{True}\NormalTok{)}
\end{Highlighting}
\end{Shaded}

The \texttt{BSS.Notebook} package provides several useful tools that are
available when working inside of a Jupyter notebook. One of these is the
plot function, that allows us to create simple x/y plots of time-series
data.

Let's grab the same record data as above and use it to make some graphs
of the data.

\begin{Shaded}
\begin{Highlighting}[]
\CommentTok{# Generate a plot of time vs temperature.}
\NormalTok{plot1 }\OperatorTok{=}\NormalTok{ BSS.Notebook.plot(process.getTime(time_series}\OperatorTok{=}\VariableTok{True}\NormalTok{), process.getTemperature(time_series}\OperatorTok{=}\VariableTok{True}\NormalTok{))}

\CommentTok{# Generate a plot of time vs energy.}
\NormalTok{plot2 }\OperatorTok{=}\NormalTok{ BSS.Notebook.plot(process.getTime(time_series}\OperatorTok{=}\VariableTok{True}\NormalTok{), process.getTotalEnergy(time_series}\OperatorTok{=}\VariableTok{True}\NormalTok{))}
\end{Highlighting}
\end{Shaded}

\begin{figure}
\centering
\includegraphics{https://github.com/michellab/BioSimSpaceTutorials/blob/dd5a24e58778af21612ade7febe5ba7fd98f9885/01_introduction/assets/03_time_series.png}
\caption{Time-series plots}
\end{figure}

(Note that, by default, the axis labels axis labels are automatically
generated from the types and units of the x and y data that are passed
to the function.)

Re-run the cell using ``CTRL+Return'' to see the graphs update as the
simulation progesses. (Occasionally, you might see a warning that the x
and y data sets are mismatched in length, this is because the data was
extracted before all records were written to disk.)

Being able to query a process in real time is an incredibly useful tool.
This could enable us to check for convergence, or spot errors in the
simulation. If you ever need to kill a running process (perhaps it was
configured incorrectly), run:

\begin{Shaded}
\begin{Highlighting}[]
\NormalTok{process.kill()}
\end{Highlighting}
\end{Shaded}

\hypertarget{visualising-the-molecular-system}{%
\subsubsection{Visualising the molecular
system}\label{visualising-the-molecular-system}}

Another useful tool that is available when working inside of a notebook
is the \texttt{View} class that can be used to visualise the molecular
system while a process is running. To create a \texttt{View} object we
must attach it to a process (or a molecular system), e.g.:

\begin{Shaded}
\begin{Highlighting}[]
\NormalTok{view }\OperatorTok{=}\NormalTok{ BSS.Notebook.View(process)}
\end{Highlighting}
\end{Shaded}

We can now visualise the system:

\begin{Shaded}
\begin{Highlighting}[]
\NormalTok{view.system()}
\end{Highlighting}
\end{Shaded}

\begin{figure}
\centering
\includegraphics{https://github.com/michellab/BioSimSpaceTutorials/blob/dd5a24e58778af21612ade7febe5ba7fd98f9885/01_introduction/assets/03_view_system.png}
\caption{Visualise the system}
\end{figure}

(If you see an empty view, try re-executing the cell.)

To only view a specific molecule:

\begin{Shaded}
\begin{Highlighting}[]
\NormalTok{view.molecule(}\DecValTok{0}\NormalTok{)}
\end{Highlighting}
\end{Shaded}

\begin{figure}
\centering
\includegraphics{https://github.com/michellab/BioSimSpaceTutorials/blob/86442df77e2ad33ae79f62e214a53af42cb320ec/01_introduction/assets/03_view_molecule.png}
\caption{Visualise a molecule}
\end{figure}

To view a list of molecules:

\begin{Shaded}
\begin{Highlighting}[]
\NormalTok{view.molecules([}\DecValTok{0}\NormalTok{, }\DecValTok{5}\NormalTok{, }\DecValTok{10}\NormalTok{])}
\end{Highlighting}
\end{Shaded}

\begin{figure}
\centering
\includegraphics{https://github.com/michellab/BioSimSpaceTutorials/blob/86442df77e2ad33ae79f62e214a53af42cb320ec/01_introduction/assets/03_view_molecules.png}
\caption{Visualise some molecules}
\end{figure}

If a particular view was of interest it can be reloaded as follows:

\begin{Shaded}
\begin{Highlighting}[]
\CommentTok{# Reload the original view.}
\NormalTok{view.}\BuiltInTok{reload}\NormalTok{(}\DecValTok{0}\NormalTok{)}
\end{Highlighting}
\end{Shaded}

\begin{figure}
\centering
\includegraphics{https://github.com/michellab/BioSimSpaceTutorials/blob/dd5a24e58778af21612ade7febe5ba7fd98f9885/01_introduction/assets/03_view_system.png}
\caption{Visualise the system}
\end{figure}

To save a specific view as a PDB file:

\begin{Shaded}
\begin{Highlighting}[]
\NormalTok{view.savePDB(}\StringTok{"my_view.pdb"}\NormalTok{, index}\OperatorTok{=}\DecValTok{0}\NormalTok{)}
\end{Highlighting}
\end{Shaded}

\hypertarget{reading-and-analysing-trajectory-data}{%
\subsubsection{Reading and analysing trajectory
data¶}\label{reading-and-analysing-trajectory-data}}

The \texttt{BSS.Trajectory} package comes with a set of tools for
reading and analysis trajectory files. Files can be loaded directly, or
if supported, can be read from a running process.

For example, to get the trajectory from the process, run:

\begin{Shaded}
\begin{Highlighting}[]
\NormalTok{traj }\OperatorTok{=}\NormalTok{ process.getTrajectory()}
\end{Highlighting}
\end{Shaded}

(If you get an error, then the trajectory file may be in the process of
being written. Simply try again.)

To get the current number of frames:

\begin{Shaded}
\begin{Highlighting}[]
\NormalTok{traj.nFrames()}
\end{Highlighting}
\end{Shaded}

\begin{verbatim}
20
\end{verbatim}

To get all of the frames as a list of \texttt{System} objects:

\begin{Shaded}
\begin{Highlighting}[]
\NormalTok{frames }\OperatorTok{=}\NormalTok{ traj.getFrames()}
\end{Highlighting}
\end{Shaded}

Specific frames can be extracted by passing a list of indices, e.g.~the
first and last:

\begin{Shaded}
\begin{Highlighting}[]
\NormalTok{frames }\OperatorTok{=}\NormalTok{ traj.getFrames([}\DecValTok{0}\NormalTok{, }\DecValTok{-1}\NormalTok{])}
\end{Highlighting}
\end{Shaded}

Like most things in BioSimSpace, the \texttt{Trajectory} class is simply
a wrapper around existing tools. Internally, trajectories are stored as
an \href{http://mdtraj.org}{MDTraj} object. This can be obtained,
allowing the user direct access to the full power of MDTraj:

\begin{Shaded}
\begin{Highlighting}[]
\NormalTok{mdtraj }\OperatorTok{=}\NormalTok{ traj.getTrajectory()}
\BuiltInTok{type}\NormalTok{(mdtraj)}
\end{Highlighting}
\end{Shaded}

\begin{verbatim}
mdtraj.core.trajectory.Trajectory
\end{verbatim}

Alternatively, a trajectory can be returned in
\href{https://www.mdanalysis.org}{MDAnalysis} format:

\begin{Shaded}
\begin{Highlighting}[]
\NormalTok{mdanalysis }\OperatorTok{=}\NormalTok{ traj.getTrajectory(}\BuiltInTok{format}\OperatorTok{=}\StringTok{"mdanalysis"}\NormalTok{)}
\BuiltInTok{type}\NormalTok{(mdanalysis)}
\end{Highlighting}
\end{Shaded}

\begin{verbatim}
MDAnalysis.core.universe.Universe
\end{verbatim}

The \texttt{Trajectory} class also provides wrappers around some basic
MDTraj analysis tools, allowing the user to compute quantities such as
the root mean squared displacement (RMSD).

Let's measure the RMSD of the alanine-dipeptide molecule with a
reference to its configuration in the first trajectory frame. To extract
the alanine-dipeptide, we search the system for a residue named ALA.
We'll also plot the RMSD for each frame of the trajectory.

\begin{Shaded}
\begin{Highlighting}[]
\CommentTok{# Search the system for a residue named ALA. Since there is a single match,}
\CommentTok{# we take the first result.}
\NormalTok{molecule }\OperatorTok{=}\NormalTok{ system.search(}\StringTok{"mol with resname ALA"}\NormalTok{)[}\DecValTok{0}\NormalTok{]}

\CommentTok{# Get the indices of the atoms in the molecule, relative to the original system.}
\NormalTok{indices }\OperatorTok{=}\NormalTok{ [system.getIndex(x) }\ControlFlowTok{for}\NormalTok{ x }\KeywordTok{in}\NormalTok{ molecule.getAtoms()]}

\CommentTok{# Compute the RMSD for each frame and plot the result.}
\NormalTok{BSS.Notebook.plot(y}\OperatorTok{=}\NormalTok{process.getTrajectory().rmsd(frame}\OperatorTok{=}\DecValTok{0}\NormalTok{, atoms}\OperatorTok{=}\NormalTok{indices),}
\NormalTok{                  xlabel}\OperatorTok{=}\StringTok{"Frame"}\NormalTok{, ylabel}\OperatorTok{=}\StringTok{"RMSD"}\NormalTok{)}
\end{Highlighting}
\end{Shaded}

\begin{figure}
\centering
\includegraphics{https://github.com/michellab/BioSimSpaceTutorials/blob/dd5a24e58778af21612ade7febe5ba7fd98f9885/01_introduction/assets/03_rmsd.png}
\caption{RMSD vs frame index}
\end{figure}

Author: Lester Hedges Email:~~ lester.hedges@bristol.ac.uk

\hypertarget{nodes-interoperable-workflow-components}{%
\section{\texorpdfstring{Nodes: \emph{Interoperable workflow
components}}{Nodes: Interoperable workflow components}}\label{nodes-interoperable-workflow-components}}

The companion notebook for this section can be found
\href{https://github.com/michellab/BioSimSpaceTutorials/blob/4844562e7d2cd0b269cead56562ec16a3dfaef7c/01_introduction/04_writing_nodes.ipynb}{here}

So far we have been working with BioSimSpace in a rather ad hoc fashion.
While this intereactive exploration is a great way of learning and
prototyping ideas, it is not a good way of producing a reproducible and
interoperable script that can be shared with others. For example, we
created processes that used specific packages such as AMBER and GROMACS.
If a user didn't have these available on their system, then the script
simply wouldn't work. We also used hard-coded paths to input files. This
means the user would have to edit the paths each time they ran the
script with different input, which would quickly become tedious.

In order to solve this problem, a core concept of BioSimSpace is the
interoperable workflow component, or \emph{node}. These are robust and
portable Python scripts that typically do a small, well-defined piece of
work. All inputs and outputs from the node are validated and the node is
written in a such a way that it is \emph{independent} of the underlying
software packages, i.e.~the same script can work with a range of
different packages. In addition, nodes are aware of the environment in
which they are run, so can be used interactively, from the command-line,
or within a workflow engine.

While it is possible to write a node directly as a Python script, we
suggest that the best way of writing one is inside of a
\href{http://jupyter.org}{Jupyter} notebook. As you've already seen, the
interactive notebook environment provides a fantastic way of prototyping
and documenting your node and will allow a user to interact with it
directly on a remote cloud server, such as
\href{https://notebook.biosimspace.org}{notebook.biosimspace.org}. The
notebook can provide a complete record of your work, inlcuding
documentation, visualisation, and graphs. When you are happy with the
node, you can download it as a regular Python script (by clicking on
\texttt{File/Download\ As/Python} in JupyterHub or
\texttt{File/Export\ Notebook\ As/Export\ Notebook\ to\ Executable\ Script}
in JupyterLab) and run it directly from the command-line on your
workstation, laptop, or on a high-performance computing cluster. Any
interactive BioSimSpace elements, such as molecular visualisations, will
simply be ignored when run this way.

\hypertarget{an-example-minimisation}{%
\subsection{An example: Minimisation}\label{an-example-minimisation}}

In the rest of the notebook you'll learn how to use BioSimSpace to write
a robust and interoperable workflow node to perform energy minimisation
on a molecular system.

As always, we'll first need to import BioSimSpace:

\begin{Shaded}
\begin{Highlighting}[]
\ImportTok{import}\NormalTok{ BioSimSpace }\ImportTok{as}\NormalTok{ BSS}
\end{Highlighting}
\end{Shaded}

We begin by creating a \texttt{Node} object. This is the core of our
molecular workflow component. It defines what it does, what input is
needed, and the output that is produced.

\begin{Shaded}
\begin{Highlighting}[]
\NormalTok{node }\OperatorTok{=}\NormalTok{ BSS.Gateway.Node(}\StringTok{"A node to perform energy minimisation and save the minimised molecular configuration to file."}\NormalTok{)}
\end{Highlighting}
\end{Shaded}

We'll now set the author and license of the node. When nodes are run the
the authorship can be queried so that people can get credit for their
work. Eventually, BioSimSpace nodes also will also contain built in
tracking information to determine how many times they are run.

\begin{Shaded}
\begin{Highlighting}[]
\NormalTok{node.addAuthor(name}\OperatorTok{=}\StringTok{"Lester Hedges"}\NormalTok{, email}\OperatorTok{=}\StringTok{"lester.hedges@bristol.ac.uk"}\NormalTok{, affiliation}\OperatorTok{=}\StringTok{"University of Bristol"}\NormalTok{)}
\NormalTok{node.setLicense(}\StringTok{"GPLv3"}\NormalTok{)}
\end{Highlighting}
\end{Shaded}

Nodes require inputs. To specify inputs we use the \texttt{BSS.Gateway}
package, which is used as a bridge between BioSimSpace and the outside
world. This will allow us to document the inputs, define their type, and
specify any constraints on their allowed values. Here we will need a set
of files that define the molecular system, and an integer that indicates
the number of minimisation steps to perform.

\begin{Shaded}
\begin{Highlighting}[]
\NormalTok{node.addInput(}\StringTok{"files"}\NormalTok{, BSS.Gateway.FileSet(}
    \BuiltInTok{help}\OperatorTok{=}\StringTok{"A set of molecular input files."}\NormalTok{)}
\NormalTok{)}

\NormalTok{node.addInput(}\StringTok{"steps"}\NormalTok{, BSS.Gateway.Integer(}
    \BuiltInTok{help}\OperatorTok{=}\StringTok{"The number of minimisation steps."}\NormalTok{,}
\NormalTok{    minimum}\OperatorTok{=}\DecValTok{0}\NormalTok{,}
\NormalTok{    maximum}\OperatorTok{=}\DecValTok{1000000}\NormalTok{,}
\NormalTok{    default}\OperatorTok{=}\DecValTok{10000}\NormalTok{)}
\NormalTok{)}

\NormalTok{node.addInput(}\StringTok{"engine"}\NormalTok{, BSS.Gateway.String(}
    \BuiltInTok{help}\OperatorTok{=}\StringTok{"The molecular dynamics engine"}\NormalTok{,}
\NormalTok{    allowed}\OperatorTok{=}\NormalTok{BSS.MD.engines(),}
\NormalTok{    default}\OperatorTok{=}\StringTok{"auto"}\NormalTok{)}
\NormalTok{)}
\end{Highlighting}
\end{Shaded}

Note that the input requirements \texttt{steps} and \texttt{engine} have
default values, so are optional.

We now need to define the output of the node. In this case we will
return a set of files representing the minimised molecular system.

\begin{Shaded}
\begin{Highlighting}[]
\NormalTok{node.addOutput(}\StringTok{"minimised"}\NormalTok{, BSS.Gateway.FileSet(}\BuiltInTok{help}\OperatorTok{=}\StringTok{"The minimised molecular system."}\NormalTok{))}
\end{Highlighting}
\end{Shaded}

When working interactively within a Jupyter notebook we need a way to
allow users to set the input requirements. The
\texttt{node.showControls} method will display a graphical user
interface (GUI), from which inputs can be set. All of the elements for
this GUI are automatically generated by the \texttt{addInput} and
\texttt{addOutput} functions above. As you'll see in the next section,
if we were to run the same node from the command-line, we would instead
get an automatically generated
\href{https://docs.python.org/3/library/argparse.html}{argparse} parser.

Note that the GUI requires active user input. All input requirements
that don't have a default value \emph{must} be set before the node can
proceed. If you try to query the node for one of the user values then an
error will be raised. For bounded integer inputs you can use a slider to
set the value, or type in the input box and press enter.

When working interactively you will typically be running on a remote
server where you won't have access to the local filesystem. In this case
you'll need to upload files for any of the \texttt{File} or
\texttt{FileSet} input requirements. The GUI below will provide buttons
that allow you to browse your own filesystem and select files. Since
Jupyter has a limit of 5MB for file transfers, we provide support for
compressed formats, such as \texttt{.zip} or \texttt{.tar.gz}. (A single
archive can contain a set of files, allowing you to set a single value
for a \texttt{FileSet} requirement.) We've provided some example input
files that can be used in the training notebooks, which are available to
download from the links below. These can then be re-uploaded using the
GUI.

AMBER:
\href{https://raw.githubusercontent.com/michellab/BioSimSpace/devel/demo/amber/ala/ala.crd}{ala.crd},
\href{https://raw.githubusercontent.com/michellab/BioSimSpace/devel/demo/amber/ala/ala.top}{ala.top}

GROMACS:
\href{https://raw.githubusercontent.com/michellab/BioSimSpace/devel/demo/gromacs/kigaki/kigaki.gro}{kigaki.gro},
\href{https://raw.githubusercontent.com/michellab/BioSimSpace/devel/demo/gromacs/kigaki/kigaki.top}{kigaki.top}

When uploading files the name of the current file(s) will replace the
\texttt{Upload} button. If you need to change the file, simply click on
the button again and choose a new file.

\begin{Shaded}
\begin{Highlighting}[]
\NormalTok{node.showControls()}
\end{Highlighting}
\end{Shaded}

\begin{figure}
\centering
\includegraphics{https://github.com/michellab/BioSimSpaceTutorials/blob/c06201e9464732df5fd64fa560779ef333f59651/01_introduction/assets/04_gui.png}
\caption{Notebook GUI}
\end{figure}

Once all requirements are set then we can acces the values using the
\texttt{node.getInput} method. The first time this is called the
\texttt{node} will automatically validate all of the input and report
the user if any errors were found.

We'll now create a molecular system using the input files uploaded by
the user. As in the previous section, we don't need to specify the
format of the files, since this is automatically determined by
BioSimSpace. (BioSimSpace has support for a wide range of formats and
can convert between many formats too.)

\begin{Shaded}
\begin{Highlighting}[]
\NormalTok{system }\OperatorTok{=}\NormalTok{ BSS.IO.readMolecules(node.getInput(}\StringTok{"files"}\NormalTok{))}
\end{Highlighting}
\end{Shaded}

As learned in the previus notebook, in order to run a minimisation we
need to define a protocol. This can be done using the
\texttt{BSS.Protocol} package. Here we will create a ``best practice''
minimisation protocol, overriding the number of steps with the input
from the user.

\begin{Shaded}
\begin{Highlighting}[]
\NormalTok{protocol }\OperatorTok{=}\NormalTok{ BSS.Protocol.Minimisation(steps}\OperatorTok{=}\NormalTok{node.getInput(}\StringTok{"steps"}\NormalTok{))}
\end{Highlighting}
\end{Shaded}

We now have everything that is required to run a minimisation. To do so,
we use the \texttt{BSS.MD} package to find an appropriate molecular
dynamics package on our current environment. What package is found will
depend upon both the system and protocol, as well as the hardware that
is available to the user. (For example, the user can choose to find
packages with GPU support.)

Note that this is different to the previous section, where we
specifically launched AMBER and GROMACS processes ourselves. This is
what makes the node interoperable, i.e.~it will work regardles of what
MD packages are installed. (As long as we find a package that supports
minimisation and supports a molecular file format to which we can
convert the input system.) By adding the optional \texttt{engine}
requirement we have also allowed the user to override the \texttt{auto}
setting if they prefer to use a specific engine.

(By default, the \texttt{run} function automatically starts the process
so it will be running as once you execute the cell below.)

\begin{Shaded}
\begin{Highlighting}[]
\NormalTok{process }\OperatorTok{=}\NormalTok{ BSS.MD.run(system, protocol, engine}\OperatorTok{=}\NormalTok{node.getInput(}\StringTok{"engine"}\NormalTok{))}
\end{Highlighting}
\end{Shaded}

We now wait for the process to finish, then check whether there were any
errors before continuing. If errors were raised, then we raise an
exception and print the last 10 lines of stdout and stderr to the user.

\begin{Shaded}
\begin{Highlighting}[]
\NormalTok{process.wait()}

\ControlFlowTok{if}\NormalTok{ process.isError():}
    \BuiltInTok{print}\NormalTok{(process.stdout(}\DecValTok{10}\NormalTok{))}
    \BuiltInTok{print}\NormalTok{(process.stdout(}\DecValTok{10}\NormalTok{))}
    \ControlFlowTok{raise} \PreprocessorTok{RuntimeError}\NormalTok{(}\StringTok{"The process exited with an error!"}\NormalTok{)}
\end{Highlighting}
\end{Shaded}

When the process has finished running we can get the minimised molecular
configuration. We will save this to file using the same format as the
original system, and set the \texttt{minimised} output requirement to
the list of file names that were written.

\begin{Shaded}
\begin{Highlighting}[]
\NormalTok{node.setOutput(}\StringTok{"minimised"}\NormalTok{,}
\NormalTok{    BSS.IO.saveMolecules(}\StringTok{"minimised"}\NormalTok{, process.getSystem(), system.fileFormat()))}
\end{Highlighting}
\end{Shaded}

Finally, we validate that the node completed succesfully. This will
check that all output requirements are satisfied and that no errors were
raised by the user. Any file outputs will be available for the user to
download as a compressed archive.

Note that the validation will fail until the cell above finishes
running.

\begin{Shaded}
\begin{Highlighting}[]
\NormalTok{node.validate()}
\end{Highlighting}
\end{Shaded}

output.zip

Once we are satisfied with our node we can choosed to download it as a
regular Python script that can be run from the command-line.

In JupyterHub, click on: \texttt{File/Download\ As/Python}\\
In JupyterLab, click on:
\texttt{File/Export\ Notebook\ As/Export\ Notebook\ to\ Executable\ Script}

That's it, you've now succesfully executed your first BioSimSpace node!

\input{LIVECOMS/01_introduction/05_running_nodes}

\subsection{Tutorial 2: Funnel Meta-Dynamics}
\hypertarget{funnel-metadynamics-tutorial}{%
\section{Funnel Metadynamics
tutorial}\label{funnel-metadynamics-tutorial}}

Funnel metadynamics (fun-metaD) is a molecular dynamics-based method
that calculates the absolute binding free energy (ABFE) between a small
organic ligand and a protein. It uses an enhanced sampling method called
metadynamics, that speeds up the molecular processes of interest by
periodically adding small amounts of bias. To best separate the bound
and unbound phases, as well as increase the rate of convergence, the
exploration of the ligand in 3D space is limited by funnel-shaped
restraints.

This tutorial aims to describe how to setup and analyse fun-metaD
simulations. The main paper that describes what fun-metaD is by
\href{https://www.pnas.org/content/110/16/6358}{Limogelli \emph{et al}
2013} but in this tutorial I will describe the
\href{https://www.ncbi.nlm.nih.gov/pmc/articles/PMC7467642/}{Rhys
\emph{et al} 2020} and
\href{https://pubs.acs.org/doi/10.1021/acs.jcim.6b00772}{Saleh \emph{et
al} 2017} implementation. The main difference between the two
implementations is the functional form of the funnel restraints: the
original fun-metaD relied on a cone and a cylinder joined to make a
funnel using a step function, while the new implementation uses a single
sigmoid function. The Limogelli implementation also requires the protein
to be realigned with a reference structure to keep the funnel strictly
in place over the binding site, which hurts the performance. The
implementation I will describe here allows the funnel to move with the
protein.

Up to now, one of the biggest drawbacks to using fun-metaD for
large-scale absolute binding free energy (ABFE) calculations was the
difficulty in setting up the simulations. It's hard to know where the
funnel should be defined, how big it needs to be, what each of the
sigmoid function parameters should be set to, along with the chore of
writting PLUMED files, where each protein and ligand system will have
slightly different atom IDs.

BioSimSpace has made this easier, automatically defining the funnel
using abstract features of the protein cavity, assigning the relevant
atom IDs, suggesting reasonable default funnel parameters and allowing
the visualisation of the funnel restraints inside a Jupyter Notebook.
All this leads to a quicker setup and much faster automation of large
ABFE screening campaigns.

By the end of this tutorial you should know: 1. The basics of what
fun-metaD does. 2. How to setup fun-metaD simulations and how to
visualise the funnel restraints. 3. How to analyse the results of a
fun-metaD simulation.

Let's get started.

\hypertarget{part-1---the-theory}{%
\subsection{Part 1 - The Theory}\label{part-1---the-theory}}

Metadynamics is an enhanced sampling method that biases a simulation
along a chosen set of reaction coordinates, or as MetaD practitioners
call them, collective variables (CVs). This bias is deposited at defined
time intervals and takes the shape of a Gaussian potential.
Investigation of drug binding should involve at least one CV, distance
from the drug molecule to the protein, where the distance between them
can be biased, causing the drug to unbind. However, that single distance
is degenerate, meaning many different configurations of the drug in 3D
space will not be described by that single distance. It also involves
the exploration of a very large volume, hindering convergence.

Fun-metaD gets around both of these problems, restricting the
exploration by using funnel-shaped restraints and reducing degeneracy by
using two CVs - `projection' and `extent'. See Figure A.

\begin{figure}
\centering
\includegraphics{figures/figure1.jpeg}
\caption{Figure1}
\end{figure}

The restraints that limit the pp.ext CV follow a sigmoid function:

\begin{figure}
\centering
\includegraphics{figures/figure2.png}
\caption{Figure1}
\end{figure}

where, S is the maximal distance from the axis, at pp.proj = i, h is the
`wall\_width', f is the `wall\_buffer', b is the `beta\_cent' (the
steepness at the inflection point), x is the `s\_cent' (the inflection
point as a value of pp.proj). The exploration along the pp.proj is
limited by the `lower\_wall' and `upper\_wall' parameters. The funnel's
radius at the narrow end is equal to `wall\_buffer'. `P0' and `Px' are
the points that define the funnel's vector. From now on I'll refer to
them as p0 and p1, respectively.

It should be obvious that there is still some degeneracy - in the plane
perpendicular to the projection axis. However, this is a good compromise
between having sufficient accuracy for describing the binding of a
ligand and the tolerable simulation slowdown of using only two CVs.

``Where should the funnel point? How big should it be at the base? Do I
need to change the position of the inflexion point? The steepness? How
long should the funnel be?''

There aren't any definite answers to any of these questions. Of course,
the funnel needs to point `out', with the narrow end in the solvent,
away from any protein residues. BSS funnel assignment code addresses
that question pretty well, most of the vectors it picks for defining the
p0 and p1 points are good enough. It's still a good idea to check, by
having a look using BSS' visualisation functionality.

As for picking the parameters for the sigmoid function - the funnel will
need to be smaller than you think. There is usually only one binding
site and the funnel should enclose only it, excluding other protein
features, by setting a small `wall\_width'. This really helps with
convergence by preventing the drug molecule from exploring irrelevant
regions in the free energy surface (FES). Other parameters don't matter
that much and the default numbers will suffice in most situations.

\hypertarget{part-2---setting-up-the-system-visualising-the-funnel-and-preparing-the-fun-metad-simulations}{%
\subsection{Part 2 - Setting up the system, visualising the funnel and
preparing the fun-metaD
simulations}\label{part-2---setting-up-the-system-visualising-the-funnel-and-preparing-the-fun-metad-simulations}}

For this part, open the \texttt{02\_bss-fun-metad-tutorial.ipynb}
notebook.

\hypertarget{part-3---analysis}{%
\subsection{Part 3 - Analysis}\label{part-3---analysis}}

Open the \texttt{03\_bss-fun-metad-analysis.ipynb} notebook.

\subsection{Tutorial 3: Steered Molecular Dynamics \& Markov State Modeling}
\subsubsection{Introduction}
The drug discovery process is mainly concerned with finding small molecules that interact with a target and have a therapeutic effect. However, a large source of failure in these endeavours has been the undruggability of targets.\cite{sMD_druggability} In these cases typical drug-like molecules cannot be used for various reasons, such as due to the shape (or absence) of the target pocket (e.g. in case of protein-protein interactions)\cite{sMD_scott-ppi-2016} or the nature of the active site (e.g. if it is highly charged).\cite{sMD_ptp-rev-druggability} An alternative to directly targeting the active site is allosteric inhibition, which is a way of affecting enzyme function at one site through binding at a different one, through a network of residue interactions \cite{sMD_Motlagh-allostery,sMD_Verkhivker-allostery} This allows the choice of pockets more suitable for small molecule drug design.

An example undruggable target is protein tyrosine phosphatase 1B (PTP1B), which has a charged active site. It is a negative regulator of insulin signalling\cite{sMD_ptp1b-diabetes} and is an attractive target for type II diabetes.\cite{sMD_Wiesman} The function of PTP1B depends on the conformation of its WPD loop, which can be closed (active) or open (inactive) (Figure \ref{fig:ptp1b}). It will be used as an example system for this tutorial.

\begin{figure}[htp]
\includegraphics[width=\linewidth]{03_steered_md/figures/open-close.png}
\caption{The WPD loop of PTP1B, in two conformations: open (yellow, PDB ID: 2HNP) and closed (red, PDB ID: 1SUG).}
\label{fig:ptp1b}
\end{figure}

Since allosteric inhibition is more complex than just physically blocking the active site, knowledge of good binding is not sufficient. An assessment of whether the binder is affecting protein function is required as well. One way of evaluating this is through populations of active and inactive conformations. Here this is achieved through Markov State Models (MSMs). \cite{sMD_Prinz} An MSM is an \textit{n $\times$ n} transition matrix that described the probability of the system transitioning to some state \textit{j} given that it is in state \textit{i}, after a given lag time $\tau$. The system is treated as memoryless, i.e. the transition probabilities do not depend on any previous states, only the current one.\cite{sMD_Husic-msm} This means model building only requires local equilibrium and can make use of multiple shorter trajectories. MSMs can thus model longer timescales without the associated computational cost.\cite{sMD_Prinz}

For the MSMs to more accurately represent statistical ensemble of protein configurations, enhanced sampling methods may be used. Steered molecular dynamics (sMD) is the focus of this tutorial. The WPD loop of PTP1B opens and closes on a $\mu$s timescale,\cite{sMD_Choy-Timescales} and therefore this transition is not observed on conventional computational timescales. sMD introduces a bias potential that is added to the Hamiltonian, thus biasing the simulation towards a specified value of a chosen collective variable (CV). This can be done via PLUMED, which is integrated in BioSimSpace.\cite{sMD_steeredMD,sMD_plumed-2} Once a larger conformational space has been explored via sMD, snapshots of various conformations can be used as starting points for equilibrium MD simulations. The trajectory data from those is then used to build an MSM (Figure \ref{fig:ensemble-protocol}).

\begin{figure}[htp]
\includegraphics[width=\linewidth]{03_steered_md/figures/ensemble-md-protocol.png}
\caption{The steps of using enhanced sampling methods to gather data for statistical analysis of protein conformation ensemble. (1) Run steered MD along some collective variable (CV); (2) Extract snapshots that evenly sample available conformational space; (3) Run equilibrium MD simulations using extracted coordinates as seeds; (4) construct an MSM using trajectory data from step 3.}
\label{fig:ensemble-protocol}
\end{figure}

\subsubsection{Running sMD using BioSimSpace}
sMD using BioSimSpace is very similar to regular production, starting with importing it and reading a parameterised and equilibrated system:
\begin{python}
import BioSimSpace as BSS
system = BSS.IO.readMolecules(["data/system.prm7", 
                                "data/system.rst7"])
\end{python}

The main requirement for sMD is to set up the CV. In this case, RMSD of all heavy atoms for residues of the WPD loop (179-185) will be used. For this, a reference structure is needed, as well as specific atom indices to be used for RMSD calculation:
\begin{python}
reference = BSS.IO.readMolecules("data/reference.pdb")
            .getMolecule(0)
rmsd_indices = []
for residue in reference.getResidues():
    if 178<=residue.index()<=184:
        for atom in residue.getAtoms():
            if atom.element()!="Hydrogen (H, 1)":
                rmsd_indices.append(atom.index())
rmsd_cv = BSS.Metadynamics.CollectiveVariable.RMSD(
                system, reference, 0, rmsd_indices)
\end{python}

One thing to note when dealing with RMSD between two different structures, is that the atoms may not be in the same order. For example, atom 1 in system in this case is a hydrogen, whereas in reference it is an oxygen. BioSimSpace takes care of this by matching up the atoms in the system to the atoms in the reference. The requirements for the reference structure are that all atoms found in reference.pdb must also exist in system. They are matched by residue number and atom name. For example, if the reference structure has an atom named CA in residue 1, there must be an equivalent in the system, and they will be mapped together.

BioSimSpace has a separate protocol for steering. To set one up, steering intervals and restraints need to be specified. Generally sMD consists of four stages (Table \ref{sMD-structure}). The end times of these stages are set:
\begin{python}
start = 0* BSS.Units.Time.nanosecond
apply_force = 4 * BSS.Units.Time.picosecond
steer = 150 * BSS.Units.Time.nanosecond
relax = 152 * BSS.Units.Time.nanosecond
\end{python}

\begin{table}
    \caption{sMD structure. Initially, the force is applied over a few picoseconds, followed by the steering (the bulk of the simulation). The force is removed at the end for system relaxation.}
    \label{sMD-structure}
    \begin{tabular}{|c|c|c|}
    \hline
       Stage  &  CV value  &  Force  \\
       \hline
        1. start  &  initial value  &  none \\
        \hline
        2. apply force  &  initial value  &  specified force  \\
        \hline
        3. steering  &  specified value  &  specified force  \\
        \hline
        4. relaxation  &  specified value  & none  \\
        \hline
    \end{tabular}
\end{table}

The length of the steering step is the most important here and will depend on the system, the steering force constant, and the magnitude of the change sMD is supposed to accomplish. The restraints specify the expected end CV values and the force constant ($\vec{s}_0$(t) and $\kappa$(t)) at each step created above. The protocol can then be created:
\begin{python}
nm = BSS.Units.Length.nanometer
restraint_1 = BSS.Metadynamics.Restraint(
                rmsd_cv.getInitialValue(), 0)
restraint_2 = BSS.Metadynamics.Restraint(
                rmsd_cv.getInitialValue(), 3500)
restraint_3 = BSS.Metadynamics.Restraint(0*nm, 3500)
restraint_4 = BSS.Metadynamics.Restraint(0*nm, 0)
protocol = BSS.Protocol.Steering(rmsd_cv, 
        [start, apply_force, steer, relax], 
        [restraint_1, restraint_2, restraint_3, restraint_4], 
        runtime=152*BSS.Units.Time.nanosecond)
\end{python}

This protocol can be used to create a process. At the moment sMD in BioSimSpace is supported with AMBER and GROMACS, and requires an installation of either of these MD engines patched with PLUMED.
\begin{python}
process = BSS.Process.Amber(system, protocol)
process.getConfig()
\end{python}
\begin{lstlisting}[columns=flexible]
['Production.',
 ' &cntrl',
...
 '  pres0=1.01325,',
 '  plumed=1,',
 '  plumedfile="plumed.dat,',
 ' /']
\end{lstlisting}

The lines plumed=1 and plumedfile="plumed.dat" are what specify that PLUMED will be used. The process can now be started to run steered MD.

\subsubsection{sMD trajectory analysis and seeded MD}
The sMD increases the number of starting conformations available for the MSM data. The next step is to save frames with various WPD loop conformations and use those as starting points for seeded MD simulations. As the sMD simulation is run, the CV values are saved to a \textbf{COLVAR} file. It can be plotted to assess whether the sMD simulation has been successful. An example is shown in Figure \ref{fig:rmsd}.

\begin{figure}[htp]
    \centering
    \includegraphics[width=\linewidth]{03_steered_md/figures/COLVAR_all.png}
    \caption{RMSD change throughout an sMD simulation. As the simulation progresses, the WPD loop RMSD gets closer and closer to the closed loop crystal structure (i.e. the loop is being closed).}
    \label{fig:rmsd}
\end{figure}

Once suitable frames have been chosen (here 100 evenly spaces snapshots are used as they sample the entire RMSD range), they can be saved as PDBs with BioSimSpace:
\begin{python}
for i, index in enumerate(frames):
    frame = BSS.Trajectory.getFrame(
    trajectory='/home/user/Documents/PTP1B/steering.nc', 
    topology = '/home/user/Documents/PTP1B/system.prm7', 
    index=int(index))
    BSS.IO.saveMolecules('data/snapshot_{i+1}',
                        frame, 'pdb')
\end{python}

These PDBs are then used as starting points for an array of 100 ns equilibrium MD simulations. As this requires a lot of computational resources, it is recommended to run this on an HPC cluster.

\subsubsection{Markov State Models}
There is a lot to consider when building MSMs, and the method is not covered in this tutorial. Here the python library \href{http://emma-project.org/latest/}{PyEMMA} was used, which has extensive examples and documentation. The exact method of building the following MSM is also provided in the jupyter notebook \textbf{03\_msm\_full.ipynb}.

The results of the data obtained from the results described above is shown in Figure \ref{fig:msm}. sMD was used to steer the WPD loop of PTP1B from open to closed and from closed to open, and 100 snapshots were extracted from each trajectory (200 in total). These were then used as starting points for 100 ns equilibrium MD simulations. The model indicates that this particular way of modelling PTP1B with peptide substrate results in catalytically active conformations 2\% of the time. Returning to the idea of allosteric inhibition, if a second model, built for a system including an allosteric binder of interest, showed a decrease in active conformation probability, it would suggest that this binder has inhibition potential. 

\begin{figure}[htp]
    \centering
    \includegraphics[width=\linewidth]{03_steered_md/figures/msm_final.png}
    \caption{The population of three metastables states as predicted by the MSM.}
    \label{fig:msm}
\end{figure}
\subsection{Tutorial 4: Free Energy Perturbation}
%TODO MERGE IN TUTORIALS AT https://github.com/michellab/bssccpbiosim2022
% [From antonia Mey, Anna Herz, Finlay Clark)]
%%%%%%%%%%%%%%%%%%%%%%%%%%%%%%%%%%%%%%%%%%%%%%%%%%%%%%%%%%%%%%%%%%%%%%%%%%%%%%%%%%%%%%%%%%%
%%%%%%%%%%%%%%%%%%%%%%%%%%%%%%%%%%%%%%%%%%%%%%%%%%%%%%%%%%%%%%%%%%%%%%%%%%%%%%%%%%%%%%%%%%%
%%%%%%%%%%%%%%%%%%%%%%%%%%%%%%%%%%%%%%%%%%%%%%%%%%%%%%%%%%%%%%%%%%%%%%%%%%%%%%%%%%%%%%%%%%%
%%%%%%%%%%%%%%%%%%%%%%%%%%%%%%%%%%%%%%%%%%%%%%%%%%%%%%%%%%%%%%%%%%%%%%%%%%%%%%%%%%%%%%%%%%%
%%%%%%%%%%%%%%%%%%%%%%%%%%%%%%%%%%%%%%%%%%%%%%%%%%%%%%%%%%%%%%%%%%%%%%%%%%%%%%%%%%%%%%%%%%%
%%%%%%%%%%%%%%%%%%%%%%%%%%.  WRITE YOUR CHAPTER CONTENTS BELOW.  %%%%%%%%%%%%%%%%%%%%%%%%%%
%%%%%%%%%%%%%%%%%%%%%%%%%%%%%%%%%%%%%%%%%%%%%%%%%%%%%%%%%%%%%%%%%%%%%%%%%%%%%%%%%%%%%%%%%%%
%%%%%%%%%%%%%%%%%%%%%%%%%%%%%%%%%%%%%%%%%%%%%%%%%%%%%%%%%%%%%%%%%%%%%%%%%%%%%%%%%%%%%%%%%%%
%%%%%%%%%%%%%%%%%%%%%%%%%%%%%%%%%%%%%%%%%%%%%%%%%%%%%%%%%%%%%%%%%%%%%%%%%%%%%%%%%%%%%%%%%%%

\subsubsection{Introduction}


Computational chemists can support structure-activity relationship
studies in medicinal chemistry by making computer models that can
predict binding affinity of ligands to proteins. One of the most popular
techniques for this is Free Energy Perturbation (FEP), which relies on
simulation alchemical transformations of ligands in a congeneric series,
simulating them both in a protein target and in just a waterbox.
Relative free energies of binding ($\Delta\Delta$G in kcal/mol) can then be computed
by simply subtracting the $\Delta$G (in protein) and the $\Delta$G (in water). Some
introductory reading is recommended. \cite{mey2020best, cournia_allen_sherman_2017, kuhn_firth-clark_tosco_mey_mackey_michel_2020}

This tutorial will outline the steps needed to:

\begin{itemize}
\item
  Select a series of transformations to simulate using LOMAP
\item
  Use BioSimSpace to set up files needed for a standard FEP run in both
  SOMD and GROMACS
\item
  Run FEP using BioSimSpace on a computing cluster
\item
  Process FEP simulation results
\item
  Compile all FEP results locally and perform data analyses
\end{itemize}

For this tutorial we will be using TYK2, a common benchmarking set in
the FEP field, first used by Schrödinger in their
\href{https://pubs.acs.org/doi/abs/10.1021/ja512751q}{2015 FEP+ paper}.

\begin{figure}[htp]
\includegraphics[width=\linewidth]{04_fep/inputs/tut_imgs/tyk2_protlig.png}
\caption{Tyrosine kinase 2 (TYK2) structure with bound ligand
(ejm\_48).}
\label{tyk2_bound_fig}
\end{figure}


Typically in FEP the goal is to predict free energies of binding for a
collection of ligands (normally 10-20). Although methods exist (such as
absolute FEP) that can predict these energies directly (i.e. $\Delta$Gbind),
these are often complicated and computationally expensive. (relative)
FEP uses a basic rule in thermodynamics that dictates that, given a
thermodynamic cycle, the net energy must always be 0. FEP allows users
to compute the $\Delta\Delta$G of binding between two ligands through this mechanism
(see figure \ref{thermodynamic_cycle_fig}).

\begin{figure}[htp]
\includegraphics[width=\linewidth]{04_fep/inputs/tut_imgs/therm_cycle.png}
\caption{Thermodynamic cycle that allows FEP practicioners to
compute relative energies of binding. Because the difference between the
vertical legs equals the difference between the horizontal legs, we can
circumvent predicting $\Delta$G$_{bind}$ directly, but instead compute $\Delta\Delta$G$_{bind}$ by
transforming between two ligands in both the solvated and bound phase.}
\label{thermodynamic_cycle_fig}
\end{figure}

Because \emph{relative} energies are calculated, pairs of ligands have to be transformed into one another in both the fee and bound phase (hence the name free energy \emph{perturbation}. Typically, smaller (i.e. fewer heavy atoms) transformations are more reliable which means
that for a ligand series it is recommended to selectively make combinations of
ligands to cover the whole series. In FEP, this is done this using
\emph{perturbation networks}, typically generated by FEP softwares.
Although generating these networks can be done by hand, it is typically
better to do it programmatically to save time and create more optimal networks
(transformation reliability does not depend just on transformation size,
but also on a series of other unfavourable moiety transformations).

\subsubsection{Setting up a FEP calculation using BioSimSpace}

BioSimSpace allows users to set up and run a FEP calculation with just a
few lines of code (and input files). First, BioSimSpace is imported and
and input structures are read:

\begin{python}
import BioSimSpace as BSS
ligand_1 = BSS.IO.readMolecules("ligand_1.mol2")[0]
ligand_2 = BSS.IO.readMolecules("ligand_2.mol2")[0]
protein = BSS.IO.readMolecules("protein.pdb")[0]
\end{python}

\noindent Input molecules are parameterised using Amber-style force fields:

\begin{python}
ligand_1 = BSS.Parameters.gaff2(ligand_1).getMolecule()
ligand_2 = BSS.Parameters.gaff2(ligand_2).getMolecule()
protein = BSS.Parameters.ff14sb(protein).getMolecule()
\end{python}

\noindent We support a variety of other force fields as well. Because one ligand is transforming to the other, they need to be
well aligned. This can be done by:

\begin{python}
atom_mapping = BSS.Align.matchAtoms(ligand_1, ligand_2)
ligand_1 = BSS.Align.rmsdAlign(ligand_1, ligand_2, 
                                     atom_mapping)
\end{python}

\noindent Now a `merged' molecule (i.e. a molecule that we can
manipulate in a way such that the endpoints are either input ligands) has to be made. Adding this merged structure into the protein structure can be done simply by addition; the complete system can then be solvated.

\begin{python}
merged = BSS.Align.merge(ligand_1, ligand_2)
system = merged + protein
system_solvated = BSS.Solvent.tip3p(molecule=system, 
                box=3*[10*BSS.Units.Length.nanometer])
\end{python}

\noindent At which point all structures necessary for a FEP run are created. A FEP protocol can be set by:

\begin{python}
protocol = BSS.Protocol.FreeEnergy()
\end{python}

\noindent Note that calling no arguments sets the default FEP protocol. BioSimSpace can set up all necessary files by:

\begin{python}
freenrg_free = BSS.FreeEnergy.Relative(solvated, 
                        protocol, work_dir="output")
freenrg_bound = BSS.FreeEnergy.Relative(system_solvated, 
                        protocol, work_dir="output")
\end{python}

\noindent Running the free and bound FEP calculations is done by:

\begin{python}
freenrg_free.run() 
freenrg_bound.run() 
\end{python}
\noindent After simulations have finished we can retrieve the free energies and subtract them to get the relative binding free energy:
\begin{python}
pmf_bound, overlap_bound = freenrg_bound.analyse()
pmf_free,  overlap_free  = freenrg_free.analyse()

free_nrg_binding = \
BSS.FreeEnergy.Relative.difference(pmf_bound, pmf_free)
\end{python}

\noindent Note that this only makes sense on a workstation with GPUs or GPU cloud resources or a GPU cluster. If the code is run as above, each $\lambda$ window simulation is run sequentially which is not very efficient.

\subsubsection{Workflow of a BioSimSpace FEP pipeline}

Because a single FEP simulation typically takes hours to run on a
single GPU (depending on settings and hardware), FEP is usually run on a
computing cluster (or HPC/ cloud service). This allows practitioners to
run many simulations at the same time, turning an FEP pipeline into a
process that takes just several days (or even fewer) instead of weeks (or
even more).

Given a protein input file and a series of ligand input files, we will
be using a Jupyter Notebook that uses LOMAP to generate a perturbation
network for us. This notebook will also write all files necessary to
further prepare our FEP simulations. Because preparing ligands and
proteins for FEP can already require some heavy computation, this will
be the first process that will run on a cluster. Then, after running and
processing the FEP outputs, we can download the results back to our
local workstation. There, the analysis notebook uses FreeEnergyAnalysis
to process FEP predictions and generate plots.

\begin{figure}[htp]
\includegraphics[width=\linewidth]{04_fep/inputs/tut_imgs/fep_pipeline.png}
\caption{Schematic of the FEP pipeline in this tutorial. Whereas blue boxes
represent notebooks run on a local machine, orange boxes represent
python scripts run sequentially on a computing cluster.}
\label{fep_pipeline_fig}
\end{figure}


\subsubsection{Generating a perturbation network to run FEP on a congeneric series of ligands}

For this step, open the jupyter notebook \textbf{setup\_fep.ipynb}. If
you would like to use your own ligands and protein, you can put these in
\texttt{inputs/ligands/} and \texttt{inputs/protein/}, respectively.

After running all the cells in this notebook, the folder
\texttt{./execution\_model/} will contain everything needed to run FEP
in parallel on your cluster. To move this folder to your cluster, you
can use for instance SCP:

\begin{lstlisting}
$ scp -r execution_model uname@cluster.address:/path/to/folder
\end{lstlisting}

\noindent \emph{Note: if a computing cluster shares its file system with the local workstation the above step is not needed.}


\subsubsection{Running FEP on a computing cluster using
BioSimSpace}

The folder \texttt{./execution\_model/} contains several scripts and
folders, of which the most important are:

\begin{itemize}
\item
  \texttt{processFEP-slurm.sh} and \texttt{processFEP-lsf.sh}: running
  either of these scripts will submit all simulation jobs (depending on the
  cluster setup). Note that there are several parameters at the top of
  these scripts that have to be set (by e.g. a system administrator)
  that will inform BioSimSpace of all relevant paths to software
  dependencies and other important matters.
\item
  \texttt{./scripts/} contains all necessary scripts to run FEP with
  BioSimSpace. Advanced users can tweak more settings in these.
\end{itemize}

If everything has been set up correctly, running:

\begin{lstlisting}
$ bash processFEP-slurm.sh
\end{lstlisting}

will start the whole FEP job submission (for SLURM clusters). First, all systems will be
prepared, then run and then analysed (see figure \ref{fep_pipeline_fig}). When jobs finish, FEP predictions will be written to \texttt{./outputs/SOMD/summary.csv} (in case of SOMD
engine). Logfiles for seeing process outputs can be found in
\texttt{./logs/}. Additionally, perturbations that were simulated
successfully will have an \emph{overlap matrix} figure saved to
\texttt{./logs/} (in case of SOMD); these can be checked to check if perturbations were likely to be reliable per simulation leg. \cite{mey2020best}

For the final analysis, only \texttt{./outputs/SOMD/summary.csv} is required. Thus, this file has to be downloaded back to the local workstation; using SCP again (from the workstation, i.e. not logged into the cluster):

\begin{lstlisting}
$ scp uname@cluster.address:/path/to/folder\ /outputs/SOMD/summary.csv .
\end{lstlisting}


\subsubsection{Analysing FEP results}

For this step, open the jupyter notebook \textbf{analyse\_fep.ipynb}.
Running cells in this notebook will generate typical FEP figures
(barplots and scatterplots); if you have missing or failed
perturbations, the script should be able to work out an optimal
prediction (although at some point with enough missing FEP predictions,
ligands will of course be missing). If you happen to have experimental
affinity values, you can validate how accurate your FEP predictions are. See figure \ref{fep_barplot_fig} for an example barplot generated from \textbf{analyse\_fep.ipynb}. Note that even though we depict per-ligand binding affinities as $\Delta$G, strictly speaking these are still quantities of $\Delta\Delta$G because they are still relative binding free energies compared to a reference! The $\Delta$G notation in this way is just a commonly applied tactic to indicate that the plots are per-ligand instead of pairwise.

\begin{figure}[htp]
\includegraphics[width=\linewidth]{04_fep/inputs/tut_imgs/fep_barplot.png}
\caption{Example barplot depicting FEP results, generated with the analysis jupyter notebook included in this tutorial.}
\label{fep_barplot_fig}
\end{figure}



% %%%% YOU CAN USE THIS BLOCK FOR PYTHON:
% \begin{python}
% import numpy as np
    
% def incmatrix(genl1,genl2):
%     M = []
%     m = len(genl1)  # comment 
%     for i in m:
%         M.append(i)
    
%     return M
% \end{python}

% %%%% YOU CAN USE THIS BLOCK FOR BASH:
% \begin{lstlisting}
% $ command -arg1 -arg2 -arg3 -arg4 -arg5 -arg6 > output.file 2> err.file
% \end{lstlisting}

% %%%% YOU CAN USE THE SAME BLOCK FOR TEXT (e.g. STDOUT):
% \begin{lstlisting}[columns=flexible]
% this is an example of a message printed to StdOut!
% \end{lstlisting}

% %%%% FOR LARGE CHUNKS OF TEXTS, USE \scriptsize{} (OR EVEN \tiny{}) TO MAKE IT FIT THE COLUMNS
% {\scriptsize
% \begin{lstlisting}[columns=flexible]

% %VERSION  VERSION_STAMP = V0001.000  DATE = 06/30/15  11:44:23                  
% %FLAG TITLE                                                                     
% %FORMAT(20a4)                                                                   
% ACE                                                                             
% %FLAG POINTERS                                                                  
% %FORMAT(10I8)                                                                   
%     1912       9    1902       9      25      11      43      24       0       0
%     2619     633       9      11      24      13      21      20      10       1
%       0       0       0       0       0       0       0       1      10       0
%       0
% %FLAG ATOM_NAME                                                                 
% %FORMAT(20a4)                                                                   
% HH31CH3 HH32HH33C   O   N   H   CA  HA  CB  HB1 HB2 HB3 C   O   N   H   CH3 HH31
% HH32HH33O   H1  H2  O   H1  H2  O   H1  H2  O   H1  H2  O   H1  H2  O   H1  H2  
% O   H1  H2  O   H1  H2  O   H1  H2  O   H1  H2  O   H1  H2  O   H1  H2  O   H1  
% H2  O   H1  H2  O   H1  H2  O   H1  H2  O   H1  H2  O   H1  H2  O   H1  H2  O   
% H1  H2  O   H1  H2  O   H1  H2  O   H1  H2  O   H1  H2  O   H1  H2  O   H1  H2  
% O   H1  H2  O   H1  H2  O   H1  H2  O   H1  H2  O   H1  H2  O   H1  H2  O   H1  
% H2  O   H1  H2  O   H1  H2  O   H1  H2  O   H1  H2  O   H1  H2  O   H1  H2  O   
% H1  H2  O   H1  H2  O   H1  H2  O   H1  H2  O   H1  H2  O   H1  H2  O   H1  H2 
% \end{lstlisting}
% }


%\section{Checklists}
%Tutorials do not necessarily require the use of a checklist as in Best Practices documents; however, they can include these if desired.
%Several useful checklist formats are available, with examples presented in \texttt{sample-document.tex} in \url{github.com/livecomsjournal/article_templates/templates}.
%One example is shown here.
%
% Here is a single-column checklist that consists of multiple sub-checklists
%\begin{Checklists}
%
%\begin{checklist}{A list}
%\textbf{Single-column checklists are also straightforward by removing the asterisk}
%\begin{itemize}
%\item First thing let's do an item which breaks across lines to see how that looks
%\item Also remember
%\item And finally
%\end{itemize}
%\end{checklist}

%\begin{checklist}{Another list}
%\textbf{This is some further description.}
%\begin{itemize}
%\item First thing
%\item Also remember
%\item And finally
%\end{itemize}
%\end{checklist}
%
%\end{Checklists}








\section{Author Contributions}
%%%%%%%%%%%%%%%%
% This section mustt describe the actual contributions of
% author. Since this is an electronic-only journal, there is
% no length limit when you describe the authors' contributions,
% so we recommend describing what they actually did rather than
% simply categorizing them in a small number of
% predefined roles as might be done in other journals.
%
% See the policies ``Policies on Authorship'' section of https://livecoms.github.io
% for more information on deciding on authorship and author order.
%%%%%%%%%%%%%%%%
LH prepared Tutorial 1, DL and LH prepared Tutorial 2, AH and LH prepared Tutorial 3, JS, LH and JM prepared Tutorial 4. 
All authors contributed to manuscript writing. Authors are listed in alphabetical order, with the exception of the first co-author.
% We suggest you preserve this comment:
For a more detailed description of author contributions,
see the GitHub issue tracking and changelog at \githubrepository.

\section{Other Contributions}
%%%%%%%%%%%%%%%
% You should include all people who have filed issues that were
% accepted into the paper, or that upon discussion altered what was in the paper.
% Multiple significant contributions might mean that the contributor
% should be moved to authorship at the discretion of the a
%
% See the policies ``Policies on Authorship'' section of https://livecoms.github.io for
% more information on deciding on authorship and author order.
%%%%%%%%%%%%%%%
Gratitude is expressed to the users of BioSimSpace who have given important feedback over the past years that have influenced the production of our tutorials and documentation. 
% We suggest you preserve this comment:
For a more detailed description of contributions from the community and others, see the GitHub issue tracking and changelog at \githubrepository.

\section{Potentially Conflicting Interests}
%%%%%%%
%Declare any potentially competing interests, financial or otherwise
%%%%%%%
JM is a member of the Scientific Advisory Board of Cresset. 

\section{Funding Information}
%%%%%%%
% Authors should acknowledge funding sources here. Reference specific grants.
%%%%%%%
JM acknowledges support from an EPSRC standard grant (EP/P022138/1) and from the University of Edinburgh UCB via a an EPSRC-Impact Acceleration Account (IAA PIII074).

\section*{Author Information}
\makeorcid

\bibliography{references}

%%%%%%%%%%%%%%%%%%%%%%%%%%%%%%%%%%%%%%%%%%%%%%%%%%%%%%%%%%%%
%%% APPENDICES
%%%%%%%%%%%%%%%%%%%%%%%%%%%%%%%%%%%%%%%%%%%%%%%%%%%%%%%%%%%%

%\appendix


\end{document}