\hypertarget{steered-md-in-biosimspace}{%
\section{Steered MD in BioSimSpace}\label{steered-md-in-biosimspace}}

Author: Adele Hardie

Email: adele.hardie@ed.ac.uk

\hypertarget{requirements}{%
\paragraph{Requirements:}\label{requirements}}

\begin{itemize}
\tightlist
\item
  BioSimSpace
\item
  numpy
\item
  pandas
\item
  matplotlib
\item
  A steered MD trajectory
\end{itemize}

The purpose of the steered MD simulation is to access conformational
space that would take a very long time (or be inaccessible altogether)
at equilibrium. To generate the data for the Markov State Model, we need
to see how the system behaves given some starting conformation. To do
this, we will be running seeded MD simulations, where a snapshot from
the sMD trajectory is used as a starting point for an equilibrium MD
simulation.

Start by importing required libraries:

\begin{Shaded}
\begin{Highlighting}[]
\ImportTok{import}\NormalTok{ pandas }\ImportTok{as}\NormalTok{ pd}
\ImportTok{import}\NormalTok{ numpy }\ImportTok{as}\NormalTok{ np}
\ImportTok{import}\NormalTok{ matplotlib.pyplot }\ImportTok{as}\NormalTok{ plt}
\ImportTok{import}\NormalTok{ BioSimSpace }\ImportTok{as}\NormalTok{ BSS}
\ImportTok{import}\NormalTok{ os}
\end{Highlighting}
\end{Shaded}

\hypertarget{plot-steering-output}{%
\subsection{Plot steering output}\label{plot-steering-output}}

PLUMED outputs a file with the CV value that was used for steering, so
we can see the progression during the simulation. The file is
automatically named \texttt{COLVAR} by BioSimSpace. Here it is loaded as
a pandas dataframe.

\begin{Shaded}
\begin{Highlighting}[]
\NormalTok{steering_output_file }\OperatorTok{=} \StringTok{'data/COLVAR'}
\NormalTok{df }\OperatorTok{=}\NormalTok{ pd.read_csv(steering_output_file, sep}\OperatorTok{=}\StringTok{' '}\NormalTok{, skipinitialspace}\OperatorTok{=}\DecValTok{1}\NormalTok{, comment}\OperatorTok{=}\StringTok{'#'}\NormalTok{, header}\OperatorTok{=}\VariableTok{None}\NormalTok{)}
\end{Highlighting}
\end{Shaded}

PLUMED outputs time in picoseconds and RMSD in nanometers. For easier
plotting, we change time to nanoseconds and RMSD to Angstrom.

\begin{Shaded}
\begin{Highlighting}[]
\NormalTok{df[}\DecValTok{0}\NormalTok{] }\OperatorTok{=}\NormalTok{ df[}\DecValTok{0}\NormalTok{]}\OperatorTok{/}\DecValTok{1000}
\NormalTok{df[}\DecValTok{1}\NormalTok{] }\OperatorTok{=}\NormalTok{ df[}\DecValTok{1}\NormalTok{]}\OperatorTok{*}\DecValTok{10}
\NormalTok{df.set_index(}\DecValTok{0}\NormalTok{, inplace}\OperatorTok{=}\VariableTok{True}\NormalTok{)}
\end{Highlighting}
\end{Shaded}

Now the RMSD change can be plotted:

\begin{Shaded}
\begin{Highlighting}[]
\NormalTok{ax }\OperatorTok{=}\NormalTok{ df[}\DecValTok{1}\NormalTok{].plot(figsize}\OperatorTok{=}\NormalTok{(}\DecValTok{12}\NormalTok{,}\DecValTok{5}\NormalTok{))}
\NormalTok{ax.set_ylabel(}\StringTok{'RMSD/$\textbackslash{}AA$'}\NormalTok{)}
\NormalTok{ax.set_xlabel(}\StringTok{'time/ns'}\NormalTok{)}
\NormalTok{ax.set_xlim(}\DecValTok{0}\NormalTok{, }\DecValTok{152}\NormalTok{)}
\end{Highlighting}
\end{Shaded}

Here the loop RMSD went down to almost 1 A. This indicates that the loop
conformation was very similar to the crystal structure of PTP1B with the
loop closed (which was used as the target) and so we can proceed with
extracting snapshots to use as seeds.

However, sMD might not work on the first try - the steering duration and
force constant used is highly dependent on each individual system. Below
is an example of a failed steering attempt:

Here steering was carried out for 80 ns only, and the force constant
used was 2500 kJ mol-1. The RMSD was decreasing as expected, but only
reached around 2 A. This was deemed insufficient and a longer steering
protocol with a larger force constant was decided upon in the end.
Ultimately this will depend on the system you are working with and what
the goal of the steering is.

\hypertarget{extract-snapshots}{%
\subsection{Extract snapshots}\label{extract-snapshots}}

In this case we will be extracting 100 evenly spaced snapshots to be
used as starting points for the seeded MD simulations.

\begin{Shaded}
\begin{Highlighting}[]
\NormalTok{snapshot_dir }\OperatorTok{=} \StringTok{'data'}
\ControlFlowTok{if} \KeywordTok{not}\NormalTok{ os.path.exists(snapshot_dir):}
\NormalTok{    os.mkdir(snapshot_dir)}
\end{Highlighting}
\end{Shaded}

Get frame indices for snapshots. Note that the end point selected is not
the end of the simulation, but the end of the steering part.

\begin{Shaded}
\begin{Highlighting}[]
\NormalTok{number_of_snapshots }\OperatorTok{=} \DecValTok{100}
\NormalTok{end }\OperatorTok{=} \DecValTok{150} \OperatorTok{/} \FloatTok{0.002}
\NormalTok{frames }\OperatorTok{=}\NormalTok{ np.linspace(}\DecValTok{0}\NormalTok{, end, number_of_snapshots, dtype}\OperatorTok{=}\BuiltInTok{int}\NormalTok{)}
\end{Highlighting}
\end{Shaded}

Check that the snapshots roughly even sample the RMSD range:

\begin{Shaded}
\begin{Highlighting}[]
\NormalTok{ax }\OperatorTok{=}\NormalTok{ df[}\DecValTok{1}\NormalTok{].iloc[frames].plot(figsize}\OperatorTok{=}\NormalTok{(}\DecValTok{12}\NormalTok{,}\DecValTok{5}\NormalTok{))}
\NormalTok{ax.set_ylabel(}\StringTok{'RMSD/$\textbackslash{}AA$'}\NormalTok{)}
\NormalTok{ax.set_xlabel(}\StringTok{'time/ns'}\NormalTok{)}
\NormalTok{ax.set_xlim(}\DecValTok{0}\NormalTok{, }\DecValTok{150}\NormalTok{)}
\end{Highlighting}
\end{Shaded}

Save each snapshot as a PDB:

\begin{Shaded}
\begin{Highlighting}[]
\ControlFlowTok{for}\NormalTok{ i, index }\KeywordTok{in} \BuiltInTok{enumerate}\NormalTok{(frames):}
\NormalTok{    frame }\OperatorTok{=}\NormalTok{ BSS.Trajectory.getFrame(trajectory}\OperatorTok{=}\StringTok{'/home/adele/Documents/PTP1B/steering.nc'}\NormalTok{, topology }\OperatorTok{=} \StringTok{'/home/adele/Documents/PTP1B/system.prm7'}\NormalTok{, index}\OperatorTok{=}\BuiltInTok{int}\NormalTok{(index))}
\NormalTok{    BSS.IO.saveMolecules(}\SpecialStringTok{f'}\SpecialCharTok{\{}\NormalTok{snapshot_dir}\SpecialCharTok{\}}\SpecialStringTok{/snapshot_}\SpecialCharTok{\{i}\OperatorTok{+}\DecValTok{1}\SpecialCharTok{\}}\SpecialStringTok{'}\NormalTok{, frame, }\StringTok{'pdb'}\NormalTok{)}
\end{Highlighting}
\end{Shaded}

These PDB files are to be used as starting points for 100 individual 100
ns simulations, starting with resolvation, minimisation and
equilibration. This is very time consuming and best done on an HPC
cluster. An \href{02_run_seededMD.py}{example script} that can be used
with an array submission is provided. Note that due to the additional
phospho residue parameters required it is specific to PTP1B with a
peptide substrate, but the \texttt{load\_system()} function can be
easily modified to work with other systems, while the rest of the script
is transferable.
