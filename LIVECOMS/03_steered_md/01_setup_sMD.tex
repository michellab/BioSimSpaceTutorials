Author: Adele Hardie

Email: adele.hardie@ed.ac.uk

\hypertarget{requirements}{%
\paragraph{Requirements:}\label{requirements}}

\begin{itemize}
\tightlist
\item
  BioSimSpace
\item
  AMBER compiled with PLUMED
\item
  An equilibrated starting system
\item
  A target structure
\end{itemize}

\hypertarget{introduction}{%
\subsection{Introduction}\label{introduction}}

Allosteric inhibition can be a useful alternative to conventional
protein targeting when the nature of the active site makes it difficult
to design binders. One of such proteins is protein tyrosine phosphatase
1B (PTP1B), which will be used as an example system for this tutorial.
The activity of PTP1B depends on the conformation of its WPD loop, which
can be open (yellow) or closed (red):

PTP1B is difficult to drug due to the charged nature of its active site,
and so is of interest for allosteric inhibition studies. However, with
allostery, knowing that a molecule binds to the protein is not enough.
It also requires assessment of whether an allosteric binder actually has
an effect on protein function. This can be assessed by Markov State
Models(MSMs). MSMs are transition matrixes that provide insight into the
statistical ensemble of protein conformations, such as what is the
probability that a protein will exist in a certain conormation.
Comparing the probability of the target protein being catalytically
active between models with and without an allsoteric binder indicates
whether or not it has potential as an inhibitor. Furthermore, since the
system is treated as memoryless, model building only requires local
equilibrium. Therefore, it can make use of shorter MD simulations,
allowing them to be run in parallel.

In order to have a more complete view of the protein ensemble, enhanced
sampling methods are used, among them steered MD (sMD) (1). It
introduces a bias potential that is added to the Hamiltonian, thus
biasing the simulation towards a specified value of a chosen collective
variable. Once the system has reached a certain conformation, those
coordinates (2) can be used as starting points for equilibrium MD
simulations (4) that can subsequently be used as data for constructing
an MSM (4). An example summary of this is shown below:

PLUMED is a library that, among other things, has enhanced sampling
algorithms. It works with multiple MD engines, including GROMACS and
AMBER. PLUMED uses a
\href{https://www.plumed.org/doc-v2.5/user-doc/html/_m_o_v_i_n_g_r_e_s_t_r_a_i_n_t.html}{moving
restraint} that is calculated as follows:

V(s⃗,t) = 1⁄2 κ(t) ( s⃗ - s⃗0(t) )2 (Eq. 1)

where s⃗0 and κ are time dependent and specified in the PLUMED input.
s⃗0 is the target CV value and κ is the force constant in kJ mol-1. The
values of both of them are set at specific steps, and linearly
interpolated in between.

This tutorial focuses on running the prerequisite simulations for MSMs
using BioSimSpace.

\hypertarget{set-up-steered-md}{%
\subsection{Set up steered MD}\label{set-up-steered-md}}

Running steered MD in BioSimSpace is very similar to regular simulations
already covered. It only requires some additional preparation for
interfacing with PLUMED, the software that takes care of biasing the
Hamiltonian.

\hypertarget{setting-up-the-system}{%
\paragraph{Setting up the system}\label{setting-up-the-system}}

We start by importing the required libraries.

\begin{Shaded}
\begin{Highlighting}[]
\ImportTok{import}\NormalTok{ BioSimSpace }\ImportTok{as}\NormalTok{ BSS}
\end{Highlighting}
\end{Shaded}

Load a system with BioSimSpace. This particular system is of PTP1B with
the WPD loop open (from PDB entry 2HNP) with a peptide substrate and has
been minimised and equilibrated.

\begin{Shaded}
\begin{Highlighting}[]
\NormalTok{system }\OperatorTok{=}\NormalTok{ BSS.IO.readMolecules([}\StringTok{'data/system.prm7'}\NormalTok{, }\StringTok{'data/system.rst7'}\NormalTok{])}
\end{Highlighting}
\end{Shaded}

\hypertarget{creating-the-cv}{%
\paragraph{Creating the CV}\label{creating-the-cv}}

A collective variable is required to run sMD, as this is the value that
will be used to calculate the biasing potential. In this case, the CV is
RMSD of the heavy atoms in the WPD loop (residues 178-184) when the WPD
loop is closed (i.e.~steering the loop from open to closed
conformation). Let's load this reference structure.

\begin{Shaded}
\begin{Highlighting}[]
\NormalTok{reference }\OperatorTok{=}\NormalTok{ BSS.IO.readMolecules(}\StringTok{'data/reference.pdb'}\NormalTok{).getMolecule(}\DecValTok{0}\NormalTok{)}
\end{Highlighting}
\end{Shaded}

Since not all of the atoms in the reference will be used to calculate
the RMSD, we check all the residues and append the appropriate atom
indices to the \texttt{rmsd\_indices} list. Here we check all the
residues instead of directly accessing the residue list in case there
are some residues missing in the structure.

\begin{Shaded}
\begin{Highlighting}[]
\NormalTok{rmsd_indices }\OperatorTok{=}\NormalTok{ []}
\ControlFlowTok{for}\NormalTok{ residue }\KeywordTok{in}\NormalTok{ reference.getResidues():}
    \ControlFlowTok{if} \DecValTok{178}\OperatorTok{<=}\NormalTok{residue.index()}\OperatorTok{<=}\DecValTok{184}\NormalTok{:}
        \ControlFlowTok{for}\NormalTok{ atom }\KeywordTok{in}\NormalTok{ residue.getAtoms():}
            \ControlFlowTok{if}\NormalTok{ atom.element()}\OperatorTok{!=}\StringTok{'Hydrogen (H, 1)'}\NormalTok{:}
\NormalTok{                rmsd_indices.append(atom.index())}
\end{Highlighting}
\end{Shaded}

Once we have our system and reference, and we know which atoms will be
used for the RMSD calculation, we can create a
\texttt{CollectiveVariable} object.

\begin{Shaded}
\begin{Highlighting}[]
\NormalTok{rmsd_cv }\OperatorTok{=}\NormalTok{ BSS.Metadynamics.CollectiveVariable.RMSD(system, reference, }\DecValTok{0}\NormalTok{, rmsd_indices)}
\end{Highlighting}
\end{Shaded}

One thing to note when dealing with RMSD between two different
structures, is that the atoms may not be in the same order. For example,
atom 1 in \texttt{system} in this case is a hydrogen, whereas in
\texttt{reference} it is an oxygen. BioSimSpace takes care of this by
matching up the atoms in the system to the atoms in the reference.

The requirements for the reference structure are that all atoms found in
\texttt{reference.pdb} must also exist in \texttt{system}. They are
matched by residue number and atom name. For example, if the reference
structure has an atom named CA in residue 1, there must be an equivalent
in the system, and they will be mapped together.

\hypertarget{setting-up-a-steered-md-protocol}{%
\paragraph{Setting up a steered MD
protocol}\label{setting-up-a-steered-md-protocol}}

To create a protocol, we need to set up the steering restraints and
schedule. As shown in equation 1, steered MD is defined by the expected
CV value and the force constant κ at some time \emph{t}. Generally sMD
has four stages:

\begin{longtable}[]{@{}lll@{}}
\toprule
Stage & Expected end CV & Force constant\tabularnewline
\midrule
\endhead
1. start & initial value & none\tabularnewline
2. apply force & initial value & specified force\tabularnewline
3. steering & target value & specified force\tabularnewline
4. relaxation & target value & none\tabularnewline
\bottomrule
\end{longtable}

Force is usually applied over a few picoseconds, and the bulk of the
simulation is used for steering, i.e.~stage 3. We need to specify the
end times for these stages:

\begin{Shaded}
\begin{Highlighting}[]
\NormalTok{start }\OperatorTok{=} \DecValTok{0}\OperatorTok{*}\NormalTok{ BSS.Units.Time.nanosecond}
\NormalTok{apply_force }\OperatorTok{=} \DecValTok{4} \OperatorTok{*}\NormalTok{ BSS.Units.Time.picosecond}
\NormalTok{steer }\OperatorTok{=} \DecValTok{150} \OperatorTok{*}\NormalTok{ BSS.Units.Time.nanosecond}
\NormalTok{relax }\OperatorTok{=} \DecValTok{152} \OperatorTok{*}\NormalTok{ BSS.Units.Time.nanosecond}
\end{Highlighting}
\end{Shaded}

The length of the steering step is the most important here and will
depend on the system, the steering force constant, and the magnitude of
the change sMD is supposed to accomplish.

Then the restraints specify the expected end CV values and the force
constant (κ(t) and s⃗0(t)) at each step created above.

\begin{Shaded}
\begin{Highlighting}[]
\NormalTok{nm }\OperatorTok{=}\NormalTok{ BSS.Units.Length.nanometer}
\NormalTok{restraint_1 }\OperatorTok{=}\NormalTok{ BSS.Metadynamics.Restraint(rmsd_cv.getInitialValue(), }\DecValTok{0}\NormalTok{)}
\NormalTok{restraint_2 }\OperatorTok{=}\NormalTok{ BSS.Metadynamics.Restraint(rmsd_cv.getInitialValue(), }\DecValTok{3500}\NormalTok{)}
\NormalTok{restraint_3 }\OperatorTok{=}\NormalTok{ BSS.Metadynamics.Restraint(}\DecValTok{0}\OperatorTok{*}\NormalTok{nm, }\DecValTok{3500}\NormalTok{)}
\NormalTok{restraint_4 }\OperatorTok{=}\NormalTok{ BSS.Metadynamics.Restraint(}\DecValTok{0}\OperatorTok{*}\NormalTok{nm, }\DecValTok{0}\NormalTok{)}
\end{Highlighting}
\end{Shaded}

In this scenario, we will be using a 3500 kJ mol-1 force constant and
our target RMSD value is 0 (as close as possible to the target
structure). These schedule steps and restraints are used to create a
steering protocol.

\begin{Shaded}
\begin{Highlighting}[]
\NormalTok{protocol }\OperatorTok{=}\NormalTok{ BSS.Protocol.Steering(rmsd_cv, [start, apply_force, steer, relax], [restraint_1, restraint_2, restraint_3, restraint_4], runtime}\OperatorTok{=}\DecValTok{152}\OperatorTok{*}\NormalTok{BSS.Units.Time.nanosecond)}
\end{Highlighting}
\end{Shaded}

\hypertarget{a-quick-look-at-gromacs}{%
\paragraph{A quick look at GROMACS}\label{a-quick-look-at-gromacs}}

We have previously created a protocol for sMD, so all that is needed is
to plug it into a GROMACS process.

\begin{Shaded}
\begin{Highlighting}[]
\NormalTok{process }\OperatorTok{=}\NormalTok{ BSS.Process.Gromacs(system, protocol)}
\end{Highlighting}
\end{Shaded}

Checking the command line arguments that will be used to run this
simulation:

\begin{Shaded}
\begin{Highlighting}[]
\NormalTok{process.getArgs()}
\NormalTok{OrderedDict([(}\StringTok{'mdrun'}\NormalTok{, }\VariableTok{True}\NormalTok{), (}\StringTok{'-v'}\NormalTok{, }\VariableTok{True}\NormalTok{), deffnm}\StringTok{', '}\NormalTok{gromacs}\StringTok{'), plumed'}\NormalTok{, }\StringTok{'plumed.dat'}\NormalTok{)])}
\end{Highlighting}
\end{Shaded}

The argument \texttt{-plumed\ plumed.dat} tells GROMACS to use PLUMED,
looking at the \texttt{plumed.dat} file for instructions. This process
can be run like any other process you have seen before. All the required
files have been created in the \texttt{process.workDir()} by
BioSimSpace.

\hypertarget{steered-md-in-amber}{%
\paragraph{Steered MD in AMBER}\label{steered-md-in-amber}}

Just as with GROMACS, we simply need to create a process in AMBER:

\begin{Shaded}
\begin{Highlighting}[]
\NormalTok{process }\OperatorTok{=}\NormalTok{ BSS.Process.Amber(system, protocol)}
\end{Highlighting}
\end{Shaded}

Check the configuration of the process:

\begin{Shaded}
\begin{Highlighting}[]
\NormalTok{process.getConfig()}
\NormalTok{[}\StringTok{'Production.'}\NormalTok{,}
 \StringTok{' &cntrl'}\NormalTok{,}
 \StringTok{'  ig=-1,'}\NormalTok{,}
 \StringTok{'  ntx=1,'}\NormalTok{,}
 \StringTok{'  ntxo=1,'}\NormalTok{,}
 \StringTok{'  ntpr=100,'}\NormalTok{,}
 \StringTok{'  ntwr=100,'}\NormalTok{,}
 \StringTok{'  ntwx=100,'}\NormalTok{,}
 \StringTok{'  irest=0,'}\NormalTok{,}
 \StringTok{'  dt=0.002,'}\NormalTok{,}
 \StringTok{'  nstlim=76000000,'}\NormalTok{,}
 \StringTok{'  ntc=2,'}\NormalTok{,}
 \StringTok{'  ntf=2,'}\NormalTok{,}
 \StringTok{'  ntt=3,'}\NormalTok{,}
 \StringTok{'  gamma_ln=2,'}\NormalTok{,}
 \StringTok{'  cut=8.0,'}\NormalTok{,}
 \StringTok{'  tempi=300.00,'}\NormalTok{,}
 \StringTok{'  temp0=300.00,'}\NormalTok{,}
 \StringTok{'  ntp=1,'}\NormalTok{,}
 \StringTok{'  pres0=1.01325,'}\NormalTok{,}
 \StringTok{'  plumed=1,'}\NormalTok{,}
 \StringTok{'  plumedfile="plumed.dat,'}\NormalTok{,}
 \StringTok{' /'}\NormalTok{]}
\end{Highlighting}
\end{Shaded}

The lines \texttt{plumed=1} and \texttt{plumedfile="plumed.dat"} are
what specify that PLUMED will be used. The process can now be started to
run steered MD.

\hypertarget{running-smd}{%
\paragraph{Running sMD}\label{running-smd}}

There are a few ways to run the simulation once it has been set up,
which will depend on what is available to you.

To run locally with this setup, simply start the process:

\begin{Shaded}
\begin{Highlighting}[]
\NormalTok{process.start()}
\end{Highlighting}
\end{Shaded}

This will start the simulation in the background. It is recommended to
set your \texttt{CUDA\_VISIBLE\_DEVICES} environment variable to an
available GPU before launching the notebook.

However, you may want to run this on a cluster rather than your own
workstation. To use the same files that we set up just now, zip them up
and copy them over. In python:

\begin{Shaded}
\begin{Highlighting}[]
\NormalTok{process.getInput()}
\end{Highlighting}
\end{Shaded}

In your terminal:

\begin{Shaded}
\begin{Highlighting}[]
\FunctionTok{scp}\NormalTok{ amber_input user@somer.cluster:/path/to/simulation/dir}
\end{Highlighting}
\end{Shaded}

Simply unzip this and submit a \texttt{pmemd.cuda} job on the cluster
available to you. Since the PLUMED usage is in the \texttt{amber.cfg}
file, no other steps are necessary.

Alternatively an \href{01_run_sMD.py}{example script} is available. It
will run the same set up steps as above on your cluster, as well as the
actual sMD simulation and will copy the output files from \texttt{/tmp}
to the local running directory. It requires a topology, equilibrated
coordinate, and reference PDB files (same as this example).
